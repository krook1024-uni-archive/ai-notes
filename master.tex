\documentclass[a4paper]{report}

\title{Mesterséges intelligencia alapjai jegyzet}
\author{Molnár Antal Albert}

\usepackage[utf8]{inputenc}
\usepackage[T1]{fontenc}
\usepackage{textcomp}
\usepackage[magyar]{babel}
\usepackage{amsmath, amsfonts, mathtools, amsthm, amssymb}
\usepackage{graphicx}
\usepackage{float}
\usepackage{titlesec}
\usepackage{algorithmicx}
\usepackage{parskip}
\usepackage{tabularx}
\usepackage{hyperref}
\usepackage[ruled,longend,linesnumbered]{algorithm2e}
\usepackage[margin=2cm]{geometry}

\usepackage{tikz}
\usetikzlibrary{positioning}
\tikzset{set/.style={draw,circle,inner sep=0pt,align=center}}

% document settings
\setlength{\parindent}{0cm}
\graphicspath{ {./figures/} }

% figure support
\usepackage{import}
\usepackage{xifthen}
\pdfminorversion=7
\usepackage{pdfpages}
\usepackage{transparent}
\newcommand{\incfig}[1]{%
    \def\svgwidth{\columnwidth}
    \import{./figures/}{#1.pdf_tex}
}

%setup alg2e
\DontPrintSemicolon
\SetKwFunction{Ftreesearch}{Fa-Kereses}
\SetKwFunction{Fgraphsearch}{Graf-Kereses}
\SetKwFunction{FACone}{AC1}
\SetKwFunction{FACthree}{AC3}
\SetKwFunction{FACfourinit}{AC4-initialize}
\SetKwFunction{FACfour}{AC4}
\SetKwFunction{Frezolval}{Rezolválás}
\SetKwFunction{FACfour}{AC4}
\SetKwProg{Fn}{function}{:}{end function}
\SetKwFor{Loop}{loop}{do}{end loop}
\SetKwInOut{Input}{input}
\SetKwInOut{Output}{output}

% Environments
\makeatother
% For box around Definition, Theorem, \ldots
\usepackage{mdframed}
\mdfsetup{skipabove=1em,skipbelow=0em}
\theoremstyle{definition}
\newmdtheoremenv[nobreak=true]{definicio}{Definíció}
\newmdtheoremenv[nobreak=true]{pelda}{Példa}
\newmdtheoremenv[nobreak=true]{megjegyzes}{Megjegyzés}
\newmdtheoremenv[nobreak=true]{lemma}{Lemma}
\newmdtheoremenv[nobreak=true]{tetel}{Tétel}
\newmdtheoremenv[nobreak=true]{informacio}{Információ}
\newmdtheoremenv[nobreak=true]{konkluzio}{Konkluzio}
\newmdtheoremenv[nobreak=true]{sejtes}{Sejtés}
\newmdtheoremenv[nobreak=true]{allitas}{Állítás}
\newmdtheoremenv[nobreak=true]{tulajdonsagok}{Tulajdonságok}

\newtheorem*{ismetles}{Ismétlés}
\newtheorem*{bizonyitas}{Bizonyítás}
\newtheorem*{gyakorlas}{Gyakorlás}
\newtheorem*{problema}{Probléma}
\newtheorem*{terminologia}{Terminológia}
\newtheorem*{spec}{Speciálisan}
\makeatletter

\DeclareMathOperator{\card}{card}
\DeclareMathOperator{\dom}{dom}
\DeclareMathOperator{\revise}{revise}
\DeclareMathOperator{\rng}{rng}

\pdfsuppresswarningpagegroup=1

\begin{document}

\maketitle

\section*{Előszó}
Ez a jegyzet a Debreceni Egyetemen a Kádek Tamás által oktatott,
\textit{INBPM0418E} tárgykódú \textit{A mesterséges intelligencia alapjai}
tárgyhoz nyújt némi segítséget a vizsgára készülő hallgatóknak.

A leírtakért semmilyen felelősséget nem tudok vállalni, hiszen még jómagam is
csak ismerkedem a mesterséges intelligencia világával.

A jegyzet nagyrészt az órákon tárgyalt könyvből\cite{nagykonyv}, valamint
Várterész Magda előadásaiból\cite{varteresz} merít.


\tableofcontents

% begin chapters
\chapter{Témakörök, melyeket mélységben ismerni kell}

\section{Ágens szemlélet}

\subsection{Az ágens fogalma}

\begin{definicio}
    Ágens.
    Egy \textbf{ágens} (agent) bármi lehet, amit úgy tekinthetünk, mint ami az
    \textbf{érzékelői} (sensors) segítségével érzékeli a \textbf{környezetét}
    (environment), és \textbf{beavatkozói} (actuators) segítségével
    megváltoztatja azt.

    Az emberi ágensnek van szeme, füle és egyéb szervei az érzékelésre, és keze,
    lába, szája és egyéb testrészei a beavatkozásra.

    A robotágens kamerákat és infravörös távolsági keresőket használ
    érzékelőként, és különféle motorokat beavatkozóként.

    A szoftverágens billentyűleütéseket, fájltartalmakat és hálózati
    adatcsomagokat fogad érzékelőinek bemeneteként, és képernyőn történő
    kijelzéssel, fájlok írásával, hálózati csomagok küldésével avatkozik be a
    környezetébe.

    Azzal az általános feltételezéssel fogunk élni, hogy minden ágens képes
    saját akcióinak érzékelésére (de nem mindig látja azok hatását).
\end{definicio}

\begin{definicio}
    Érzékelés.
    Az érzékelés (percept) fogalmat használjuk az ágens érzékelő bemeneteinek
    leírására egy tetszőleges pillanatban.
\end{definicio}

\begin{definicio}
    Ágens érzékelési sorozata.
    Egy ágens érzékelési sorozata (percept sequence) az ágens érzékeléseinek
    teljes története, minden, amit az ágens valaha is érzékelt. Általánosságban,
    egy adott pillanatban egy ágens cselekvése az addig megfigyelt teljes
    érzékelési sorozatától függhet. Ha az összes lehetséges érzékelési
    sorozathoz meg tudjuk határozni az ágens lehetséges cselekvéseit, akkor
    lényegében mindent elmondtunk az ágensről.
\end{definicio}

\begin{definicio}
    Ágensfüggvény.
    Matematikailag megfogalmazva azt mondhatjuk, hogy az ágens viselkedését az
    ágensfüggvény (agent function) írja le, ami az adott érzékelési sorozatot
    egy cselekvésre képezi le.
\end{definicio}

\begin{definicio}
    Ágensprogram.
    Egy mesterséges ágens belsejében az ágensfüggvényt egy ágensprogram (agent
    program) valósítja meg.
\end{definicio}

\begin{megjegyzes}
    Különbség ágensfüggvény és ágensprogram között.
    Az ágensfüggvény egy absztrakt matematikai leírás, az ágensprogram egy
    konkrét implementáció, amely az ágens architektúráján működik.
\end{megjegyzes}

\subsection{Az ágens jellemzése (teljesítmény, környezet, érzékelők, beavatkozók}

\begin{definicio}
    \textbf{TKBÉ} (PEAS) leírás.
    Egy ágens tervezése során az első lépésnek mindig a feladatkörnyezet lehető
    legteljesebb meghatározásának kell lennie.

            \begin{tabularx}{\textwidth}{|X|X|X|X|X|}
            \hline
            \textbf{Ágenstípus} &
            \textbf{Tel-jesítmény-mérték (Performance)} &
            \textbf{Környezet (Environment)} &
            \textbf{Beavatkozók (Actuators)} &
            \textbf{Érzékelők (Sensors)}
            \\\hline
            Taxisofőr &
            Biztonságos, gyors, törvényes, kényelmes utazás, maximum haszon &
            Utak, egyéb forgalom, gyalogosok, ügyfelek &
            Kormány, gáz, fék, index, kürt, kijelző &
            Kamerák, hangradas, sebességmérő, GPS, kilométeróra, motorérzékelők,
            billentyűzet
            \\\hline
        \end{tabularx}
\end{definicio}

\begin{definicio}
    Teljesen megfigyelhető környezet.

    Ha az ágens szenzorai minden pillanatban hozzáférést nyújtanak a környezet
    teljes állapotához, akkor azt mondjuk, hogy a környezet teljesen
    megfigyelhető.
\end{definicio}

\begin{definicio}
    Determinisztikus (deterministic) vagy sztochasztikus (stochastic).

    Amennyiben a környezet következő állapotát jelenlegi állapota és az ágens
    által végrehajtott cselekvés teljesen meghatározza, akkor azt mondjuk, hogy
    a környezet determinisztikus, egyébként sztochasztikus.
\end{definicio}

\begin{definicio}
    Epizódszerű (episodic) vagy sorozatszerű (sequential) környezet.

    Epizódszerű környezetben az ágens tapasztalata elemi "epizódokra" bontható.
    Minden egyes epizód az ágens észleléseiből és egy cselekvéséből áll. Nagyon
    fontos, hogy a következő epizód nem függ az előzőben végrehajtott
    cselekvésektől.  Epizódszerű környezetekben az egyes epizódokban az akció
    kiválasztása csak az aktuális epizódtól függ.
\end{definicio}

\begin{definicio}
    Statikus (static) vagy dinamikus (dynamic) környezet.

    Ha a környezet megváltozhat, amíg az ágens gondolkodik, akkor azt mondjuk,
    hogy a környezet az ágens számára dinamikus; egyébként statikus.
\end{definicio}

\begin{definicio}
    Diszkrét (discrete) vagy folytonos (continuous) környezet.

    A diszkrét/folytonos felosztás alkalmazható a környezet állapotára, az
    időkezelés módjára, az ágens észleléseire, valamint cselekvéseire. Például
    egy diszkrét állapotú környezet, mint amilyen a sakkjáték, véges számú
    különálló állapottal rendelkezik. A sakkban szintén diszkrét az akciók és
    cselekvések halmaza. A taxivezetés folytonos állapotú és idejű probléma: a
    sebesség, a taxi és más járművek helye folytonos értékek egy tartományát
    járja végig a folytonos időben.
\end{definicio}

\begin{definicio}
    Egyágenses (single agent) vagy többágenses (multiagent) környezet.

    Az egyágenses és többágenses környezetek közötti különbségtétel egyszerűnek
    tűnhet. Például a keresztrejtvényt megfejtő ágens önmagában nyilvánvalóan
    egyágenses környezetben van, míg egy sakkozó ágens egy kétágensesben.
    Vannak azonban kényes kérdések. Először is: leírtuk azt, hogy egy entitás
    hogyan tekinthető ágensnek, ugyanakkor nem magyaráztuk meg, mely entitások
    tekintendők ágensnek. Egy $A$ ágensnek (például a taxisofőrnek) egy $B$
    objektumot (egy másik járművet) ágensnek kell tekintenie, vagy egyszerűen
    egy sztochasztikusan viselkedő dolognak, a tengerparti hullámokhoz vagy a
    szélben szálló falevelekhez hasonlatosan? A választás kulcsa az, hogy vajon
    $B$ viselkedése legjobban egy $A$ viselkedésétől függő teljesítménymérték
    maximalizálásával írhatóe le. Például a sakkban a $B$ ellenfél saját
    teljesítménymértékét próbálja maximalizálni, amely – a sakk szabályainak
    következtében – A teljesítménymértékét minimalizálja. Így a sakk egy
    versengő (\textbf{competitive}) többágenses környezet. Másrészről, a taxi
    vezetési környezetben az ütközések elkerülése az összes ágens
    teljesítménymértékét maximálja, így az részben kooperatív
    (\textbf{cooperative}) többágenses környezet.  Emellett részben versengő
    is, hiszen például csak egy autó tud egy parkolóhelyet elfoglalni. A
    többágenses környezetekben felmerülő ágenstervezési problémák gyakran
    egészen mások, mint egyágenses környezetekben. Többágenses környezetekben
    például a kommunikáció (communication) gyakran racionális viselkedésként
    bukkan fel; egyes részlegesen megfigyelhető versengő környezetekben a
    sztochasztikus viselkedés racionális, hiszen így elkerülhetők a
    megjósolhatóság csapdái.
\end{definicio}

\section{Állapottér reprezentáció}

\subsection{Az állapottér fogalma}

Az állapottér-reprezentáció az egyik leggyakrabban használt reprezentációs mód
ami egy probléma formális megadására szolgál.

\begin{definicio}
    Probléma.
    Egy \textbf{probléma} (problem) formális megragadásához az alábbi négy
    komponensre van szükség:

    \begin{itemize}
        \item \textbf{kiinduló állapot}:
            amiből az ágens kezdi a cselekvéseit,
        \item \textbf{cselekvések halmaza}:
            ágens rendelkezésére álló lehetséges cselekvések,
        \item \textbf{állapotátmenet-függvény}:
            visszaadja a rendezett $\langle$ cselekvés, utódállapot $\rangle$
            párok halmazát (Egy alternatív megfogalmazás az \textbf{operátor}ok
            egy halmaza, amelyeket egy állapotra alkalmazva lehet az
            utódállapotokat generálni.) lásd még:
            \textit{\ref{operator-alkalmazasi}. Definíció}
        \item \textbf{állapottér}: A kezdeti állapot és az
            állapotátmenet-függvény együttesen implicit módon definiálják a
            probléma \textbf{állapotteré}t: azon állapotok halmazát, amelyek a
            kiinduló állapotból elérhetők.
    \end{itemize}
\end{definicio}

\begin{definicio}\label{operator-alkalmazasi}
    Operátor alkalmazási előfeltétel teszt.
    Ahhoz, hogy meggyőződjünk arról, hogy egy adott állapotra egy adott
    operátor alkalmazható, előbb meg kell vizsgálnunk, hogy az állapotra
    alkalmazható-e az operátor. Ezt a vizsgálódást operátor alkalmazási
    előfeltétel tesztnek nevezzük.
\end{definicio}

\begin{definicio}
    Célteszt.
    A célteszt ellenőrzi a célfeltételek teljesülését egy adott állapotban.
\end{definicio}

\begin{definicio}
    Állapottér-reprezentáció.
    Állapottér-reprezentáció alatt egy formális négyest értünk: \[
        \left< \mathcal{A}, k, \mathcal{C}, \mathcal{O} \right>
    ,\] ahol \begin{itemize}
    \item $\mathcal{A}$: az állapotok halmaza, $\mathcal{A} \neq \varnothing$
    \item $k \in \mathcal{A}$: a kezdőállapot
    \item $\mathcal{C} \subset \mathcal{A}$: a célállapotok halmaza
    \item $\mathcal{O}$: az operátorok halmaza, ahol $\mathcal{O} \neq \varnothing$
        \begin{itemize}
            \item $\forall o \in \mathcal{O}$ egy operátor, azaz egy $o :
                \text{Dom}(o) \to \mathcal{A}$ függvény, ahol $\text{Dom}(o) =
                \{a | a \not\in C \land \text{előfeltétel}(o,a)\}$
        \end{itemize}
    \end{itemize}
\end{definicio}

\begin{definicio}
    Közvetlen elérhetőség.
    Az $a \in \mathcal{A}$ állapotból az $a' \in \mathcal{A}$ állapot
    \textbf{közvetlenül elérhető}, ha van olyan $o_l \in \mathcal{O}$,
    amely esetén \[
        a \in \dom(o_l) \text{  és  } o_l(a) = a'
    \]
    és ezt $a \Rightarrow a'$ alakban jelöljük.
\end{definicio}

\begin{definicio}
    Elérhetőség.
    Az $a \in \mathcal{A}$ állapotból az $a' \in \mathcal{A}$ állapot
    \textbf{elérhető} ($a \Rightarrow^* a'$), ha
    \begin{itemize}
        \item $a=a'$, vagy
        \item van olyan  $a_1, a_2, \ldots, a_k$ állapotsorozat, hogy
            $a_1=a, a_k=a'$ továbbá \[
                a_i \Rightarrow a_{i+1} \text{ minden }
                i \in \{1, \ldots, k-1\} \text{ esetén}
            .\]
    \end{itemize}
\end{definicio}

\subsection{Az állapottérgráf}

\begin{definicio}
    Állapottérgráf.
    Az állapottér egy gráfot alkot,
    amelynek csomópontjai az állapotok és a csomópontok közötti élek a
    cselekvések.
\end{definicio}

\begin{definicio}
    Állapottér útja.
    Az \textbf{állapottér egy útja} az állapotok egy sorozata, amely
    állapotokat a cselekvések egy sorozata köt össze.
\end{definicio}

\begin{definicio}
    Állapottér-reprezentációs gráf bonyolultsága.  Egy állapottér-reprezentált
    probléma megoldásának sikerét jelentősen befolyásolja a reprezentációs gráf
    bonyolultsága:

    \begin{itemize}
        \item a csúcsok száma,
        \item az egy csúcsból kiinduló élek száma,
        \item a hurkok és körök száma és hossza.
    \end{itemize}

    Ezért célszerű minden lehetséges egyszerűsítést végrehajtani. Lehetséges
    egyszerűsítések:

    \begin{itemize}
        \item a csúcsok számának csökkentése — ügyes reprezentációval az
            állapottér kisebb méretű lehet;
        \item az egy csúcsból kiinduló élek számának csökkentése — az
            operátorok értelmezési tartományának alkalmas megválasztásával
            érhető el;
        \item a reprezentációs gráf fává alakítása — a hurkokat, illetve
            köröket „kiegyenesítjük”
    \end{itemize}
\end{definicio}

\subsubsection{Az állapottérgráf jellemzése}

$b$: Az {\bf elágazási tényező} (branching factor) a tetszőleges
állapotból közvetlenül elérhető állapotok maximális száma. \[
\max \{\card \{b:a \Rightarrow b\} : a \in \mathcal{A})\}
.\]

$d$ : A {\bf legsekélyebb megoldás} a legkevesebb operátoralkalmazás
segítségével elérhető célállapot eléréshez szükséges operátoralkalmazások száma.
(A legrövidebb megoldás hossza. A célállapotok minimális mélysége.)
A legkisebb olyan $i$ amely esetén van olyan állapotsorozat, hogy \[
    a_0 \Rightarrow a_1, \quad a_1 \Rightarrow a_2 \quad \ldots \quad a_{i-1}
    \Rightarrow a_i,\quad a_i \in \mathcal{C}
.\]

$m$ : A {\bf csomópontok maximális mélysége}. A legnagyobb olyan $i$, amely esetén
van olyan állapotsorozat, hogy \[
    a_0 \Rightarrow a_1,\quad a_1 \Rightarrow a_2 \quad \ldots \quad a_{i-1}
    \Rightarrow a_i
.\]

% TODO

\subsection{Költség és heurisztika fogalmak}

\begin{definicio}
    Lépésköltség.
    Az $x$ állapotból az $y$ állapotba vezető $cs$ cselekvés
    \textbf{lépésköltség}e (step cost) legyen $lk(x, cs, y)$.  Tételezzük fel,
    hogy a lépésköltség nemnegatív.
\end{definicio}

\begin{definicio}
    Útköltség-függvény.
    Egy \textbf{útköltség-függvény}, egy olyan függvény amely az állapottér
    minden útjához hozzárendel egy költséget.
\end{definicio}

\begin{definicio}
    Megoldás.
    Egy út, amely a kiinduló állapotból egy célállapotba vezet.
\end{definicio}

\begin{definicio}
    Optimális megoldás.
    A legkisebb útköltségű megoldás.
\end{definicio}

\begin{definicio}
    Heurisztikus függvény.
    Egy $\left<\mathcal{A}, k, \mathcal{C}, \mathcal{O} \right>$
    állapottér-reprezentációhoz megadott heurisztika egy olyan \[
        h : \mathcal{A} \mapsto \mathbb{N} \text{ függvény, melyre }
        \forall c \in \mathcal{C} h(c) = 0
    .\]

    A heurisztika nem más mint egy becslés amely megmondja egy csúcsra nézve,
    hogy melyik gyermeke felé induljon tovább a keresésben. A heurisztikus
    kereső nem teljesen megbízható mivel a részfák még nincsenek legenerálva,
    azaz nem lehetünk biztosak benne.

    Egyes heurisztikákat perfekt heurisztikának nevezünk és egy ilyen
    heurisztika legfeljebb $1+n*d$ csúcsot generál le, azaz ilyen heurisztika
    eseté a csúcsok száma már csak lineáris függvénye a megoldás hosszának.(
    Perfekt heurisztika egyébként nem létezik, mert ha lenne akkor már előre
    ismernénk a megoldást - nem lenne értelme keresni.)
\end{definicio}

\section{Megoldáskereső algoritmusok}

\subsection{Fakereső algoritmusok}

\subsection{Gráfkereső algoritmusok}

\subsection{Szélességi kereső}

\subsection{Mélységi kereső}

\subsection{Visszalépéses kereső}

\subsection{Egyenletes költségű (optimális kereső)}

\subsection{Legjobbat először kereső}

\subsection{Az A$^*$ algoritmus}

\section{Kétszemélyes játékok}

\subsection{A játékok reprezentációja}

A továbbiakban a
\begin{itemize}
    \item teljesen megfigyelhető,
    \item véges,
    \item determinisztikus,
    \item kétszemélyes,
    \item zérusösszegű
\end{itemize}
játékokkal foglalkozunk.

\begin{definicio}
    Játék reprezentációja.
    Egy játék reprezentációja megadható a
    \[
        \langle
        \mathcal{B},
        b_0,
        \mathcal{J},
        \mathcal{V},
        \hat{v},
        \mathcal{L}
        \rangle
    \]
    rendezett hatossal, ahol:
    \begin{itemize}
        \item $\mathcal{B}$: a játékállások halmaza,
        \item $b_0$: a kezdőállás, ahol $b_0 \in \mathcal{B}$,
        \item $\mathcal{J}$: a játékosok halmaza, ahol $\card \mathcal{J} = 2$,
        \item $\mathcal{V}$: a végállások halmaza, ahol $\mathcal{V} \subseteq
            \mathcal{B}$,
        \item $\hat{v}$: egy $\mathcal{V} \to \{-1, 0, 1\} $
            függvény,
        \item $\mathcal{L}$: a lépések halmaza.
    \end{itemize}
\end{definicio}

\begin{definicio}
    Nyertes.
    Végállásban a $\hat{v}$ függvény határozza meg, hogy melyik
    játékos nyert:
     \[
        \hat{v}(b) =
        \begin{cases}
            1 & \text{ha $b$ állásban a következő játékos nyer} \\
            0 & \text{ha $b$ állásban a következő játékos veszít} \\
            -1 & \text{egyébként (döntetlen esetén)}
        \end{cases}
    .\]
\end{definicio}

\begin{definicio}
    Lépés.
    Minden $l \in \mathcal{L}$ egy $\mathcal{B} \to \mathcal{B}$
    parciális függvény.

\end{definicio}

\begin{definicio}
    Játék állapottere.
    Legyen $\langle \mathcal{B}, b_0, \mathcal{J}, \mathcal{V}, \hat{v},
    \mathcal{L} \rangle$ egy játék reprezentációja. Ekkor a \textbf{játék
    állapottere} $\langle \mathcal{A}, a_0, \mathcal{C}, \mathcal{O} \rangle$
    definiálható a következőképpen:
     \begin{itemize}
         \item $\mathcal{A} = \mathcal{B} \times \mathcal{J}$;
         \item $a_0 = \langle b_0, j_0 \rangle$, ahol $j_0 \in \mathcal{J}$ a kezdőjátékos;
         \item $\mathcal{C} = \{\langle b, j \rangle : \langle b, j \rangle \in A \land b \in \mathcal{V}\} $
         \item $\mathcal{O} = \{o_l : l \in \mathcal{L}\}$, ahol $o_l :
             \mathcal{A} \to \mathcal{A}$, úgy hogy
             \begin{itemize}
                 \item $\dom(o_l) = \{ \langle b, j \rangle : b \in \dom(l)\}$
                 \item $o_l(\langle b, j \rangle) = \langle l(b), j'\rangle$
                 \item  $j' \in \left(\mathcal{J} \setminus  \{j\}
                     \right) $
             \end{itemize}

     \end{itemize}
\end{definicio}

\begin{definicio}
    Közvetlen elérhetőség.
    Az $a \in \mathcal{A}$ állapotból az $a' \in \mathcal{A}$ állapot
    \textbf{közvetlenül elérhető}, ha van olyan $o_l \in \mathcal{O}$,
    amely esetén \[
        a \in \dom(o_l) \text{  és  } o_l(a) = a'
    \]
    és ezt $a \Rightarrow a'$ alakban jelöljük.
\end{definicio}

\begin{definicio}
    Elérhetőség.
    $Az a \in \mathcal{A}$ állapot az $a' \in \mathcal{A}$ állapot
    \textbf{elérhető} ($a \Rightarrow^* a'$), ha
    \begin{itemize}
        \item $a=a'$, vagy
        \item van olyan  $a_1, a_2, \ldots, a_k$ állapotsorozat, hogy
            $a_1=a, a_k=a'$ továbbá \[
                a_i \Rightarrow a_{i+1} \text{ minden }
                i \in \{1, \ldots, k-1\} \text{ esetén}
            .\]
    \end{itemize}
\end{definicio}

\subsection{A játékfa}

\begin{definicio}
    Játékfa.
    Legyen $\langle \mathcal{B}, b_0, \mathcal{J}, \mathcal{V}, \hat{v},
    \mathcal{L} \rangle$ egy játék reprezentációja és $j_0 \in \mathcal{J}$ a
    kezdőjátékos. Ekkor a játékfa olyan fa, melynek csúcsaihoz a játék
    állapotait rendeljük:
    \begin{itemize}
        \item a fa gyökere a $\langle a_0, j_0 \rangle$ állapottal címkézett csúcs,
        \item a fa levélelemei olyan $\langle b, j \rangle$ állapottal
            címkézett csúcsok, ahol  $b \in \mathcal{V}$
        \item a fa $\langle b, j \rangle$ állapottal címkézett nem levélcsúcsának
            gyermekeit olyan $\langle b', j' \rangle$ címkéjű csúcsok alkotják, ahol
            $\langle b, j \rangle \Rightarrow \langle b', j' \rangle$
    \end{itemize}
\end{definicio}

\megjegyzes{A játékfa a játék állapotterének gráfját fává egyenesíti ki,
egy-egy állapot a fában több csúcs címkéjeként is szerepelhet.}

\subsection{Nyerő stratégia}

\begin{definicio}
    Játszma.
    Egy {\bf játszma} egy olyan lépéssorozat, amely a $\left<b_0, j_0 \right>$
    kezdőállapotból valamely $\left<b,j \right> \in \mathcal{C}$ célállapotba
    vezet. (Egy a gyökértől valamely levélcsúcsig vezető út a játékfában).
\end{definicio}

\begin{definicio}
    Stratégia.
    Játékterv, ami minden olyan állásban, amikor ő következik, megmondja a
    játékosnak, hogy mit lépjen.

    A $j\in\mathcal{J}$ játékos {\bf stratégiája} olyan $S_j : \{\left<b,j
    \right> : \left<b,j \right> \in \mathcal{A}\} \to \mathcal{O}$ leképezés
    (döntési terv), amely $j$ számára előírja, hogy a játék azon állapotaiban,
    ahol  $j$ a lépni következő játékos, a megtehető lépések közül melyiket
    lépje meg.
\end{definicio}

\begin{definicio}
    Nyerő stratégia.

    A $j \in \mathcal{J}$ játékos egy $S_j$ stratégiája {\bf nyerő stratégia},
    ha minden a stratégia mellett lejátszható játszmában $j$ nyer.

    Ha döntetlen állhat elő -- $0 \not\in \rng\left( \hat{v} \right)$ --,
    akkor az egyik játékos rendelkezik nyerő stratégiával. A nyerő
    stratégiával rendelkező játékos $w(a_0, j_0)$: \[
        w(b,j) = \begin{cases}
            j & \text{ ha } b \in \mathcal{V} \text{ és } \hat{v}(b) = 1 \\
            j' & \text{ ha } b \in \mathcal{V} \text{ és } \hat{v}(b) = -1 \\
            j & \text{ ha van olyan} \left<b', j' \right> \text{ hogy }
            \left<b, j \right> \rightarrow \left<b', j' \right> és
            w(b', j') = j\\
            j' & \text{ egyébként,}
        \end{cases}
    \]
    ahol $j' \in (\mathcal{J} \setminus \{j\})$.
\end{definicio}

\begin{megjegyzes}
    Ha a játékban van döntetlen végállás, akkor az egyik játékosnak van
    garantáltan nem vesztő stratégiája.
\end{megjegyzes}

\section{Lépésajánló algoritmusok}

\begin{definicio}
    Hasznosságfügvény.

    A {\bf hasznosságfügvény} egy becslés, de elávrjuk, hogy végállásban pontos
    legyen. Formálisan: \[
        h  : \mathcal{A} \to [-m, m] \text{ ahol } m \in \mathbb{R}^+
        \text{ továbbá ha } b \in \mathcal{V} \text{ akkor }
        h(b,j) = m \cdot \hat{v}(b)
    .\]

    Ha $h(b,j) > h(b', j)$ akkor a $b$ állás a $j$ (lépni következő) játékos
    számára kedvezőbb, mint a $b'$ állás.

    Legyen $h$ egy hasznosságfüggvény, ekkor $h_i$ az $i \in \mathcal{J}$ játékos
    hasznosságfüggvénye, ha \[
        h_i(a,j) = \begin{cases}
            h(a,j) & \text{ ha } i = j \\
            -h(a,j) & \text{ egyébként.}
        \end{cases}
    \]
\end{definicio}

\subsection{MinMax módszer}

Az $b \in \mathcal{B}$ állás hasznossága $t \in \mathcal{J}$ támogatott játékos
számára ha $j \in \mathcal{J}$ játékos következik lépni, $k \in \mathbb{N}$
mélységi korláttal rekurzívan a következőképpen számítható:
\[
    f(b, j, t, k) =
    \begin{cases}
        h_t(b,j) & \text{ ha } b \in \mathcal{V} ,\\
        h_t(b,j) & \text{ ha } k = 0 ,\\
        \min \{ f(b', j', t, k-1) : \left<b, j \right> \Rightarrow \left<b', j' \right>\}
        & \text{ ha } t \neq j ,\\
        \max \{ f(b', j', t, k-1) : \left<b, j \right> \Rightarrow \left<b', j' \right>\}
        & \text{ ha } t = j.
    \end{cases}
\]

\subsection{NegaMax módszer}

Az $b \in \mathcal{B}$ állás hasznossága $j \in \mathcal{J}$ (lépni következő)
játékos számára $k \in \mathbb{N}$ mélységi korláttal rekurzívan a következőképpen
számítható: \[
    \hat{f} (b, j, k) = \begin{cases}
        h(b,j) & \text{ ha } b \in \mathcal{V},\\
        h(b,j) & \text{ ha } k = 0,\\
        max \{
            -\hat{f}(b', j', k-1) : \left<b,j \right> \Rightarrow \left<b',j' \right>
        \}
               & \text{ egyébként.}
    \end{cases}
\]

\subsection{Alfa-béta nyesés}

A gyakorlatban a MinMax algoritmus javított változatát szokás használni, az
alfa-béta nyesést.

\begin{definicio}
    Az ajánlott lépés.  A $b \in \mathcal{B}$ állásban {\bf ajánlott lépés} a
    $j \in \mathcal{J}$
    (lépni következő) játékos számára olyan $l \in \mathcal{L}$ lépés, amely esetén
    \begin{itemize}
        \item $b \in \dom l$,
        \item $o_l\left(\left<b,j \right> \right) = \left<b', j' \right>$, és
        \item $f(b,j,j,k) = f(b', j', j, k-1)$.
    \end{itemize}
\end{definicio}

\section{Élkonzisztencia algoritmusok}

\begin{definicio}
    Legyen $\mathcal{V}$ változók egy tetszőleges halmaza, $\mathcal{D}$ pedig
    egy olyan leképezés, amely minden $x \in \mathcal{V}$ változóhoz egy
    $\mathcal{D}(x)$-el jelölt halmazt (tartományt) rendel.

    Tetszőleges két különböző $x \in \mathcal{V}$ és $y \in \mathcal{V}$ változó
    közti bináris kényszer egy $\mathcal{D}(x)$ és $\mathcal{D}(y)$ feletti
    bináris reláció: \[
        R_{x,y} \subseteq  \mathcal{D}(x) \times \mathcal{D}(y)
    .\]

    Jelöljük $R^{-1}_{x,y}$ formában az $R_{x,y}$ reláció inverzét: \[
        R^{-1}_{x,y} = \{
            (b,a) : (a,b) \in R_{x,y}
        \}
    .\]
\end{definicio}

\begin{definicio}
    Véges, bináris kényszerkielégítési probléma.
    Egy {\bf véges, bináris kényszerkielégítése probléma} alatt olyan
    $\left<\mathcal{V}, \mathcal{D}, \mathcal{C} \right>$ rendezett hármast
    értünk, ahol
    \begin{itemize}
        \item $\mathcal{V}$ változók egy véges, nemüres halmaza
        \item $\mathcal{D}(x)$ minden $x \in \mathcal{V}$ változó esetén egy
            véges halmaz(tartomány)
        \item $\mathcal{C}$ a $\mathcal{V}$-beli változók feletti bináris
            kényszerek halmaza, úgy hogy
            \begin{itemize}
                \item ha $R_{x_y} \in \mathcal{C}$ és $R'_{x,y} \in
                    \mathcal{C}$ akkor $R_{x,y} = R'_{x,y}$.
                \item ha $R_{x_y} \in \mathcal{C}$ akkor $R_{y,x} \in
                    \mathcal{C}$ és $R_{y,x} = R^{-1}_{x,y}$.
            \end{itemize}
    \end{itemize}
\end{definicio}

\begin{definicio}
    Gráf reprezentáció.
    Egy $\left<\mathcal{V}, \mathcal{D}, \mathcal{C} \right>$(nem feltétlenül
    véges) {\bf bináris kényszerkielégítési probléma gráf reprezentációja}
    egy olyan $\left<N, E \right>$ pár, ahol:
    \begin{itemize}
        \item $N = \{n_v : v \in \mathcal{V}\}$ alkotja a gráf csúcsait,
        \item $E = \{
                (n_x, n_y) : R_{x,y} \in \mathcal{C}
            \}$ alkotja a gráf irányított éleit.
    \end{itemize}

    \begin{megjegyzes}
        Mivel $R_{x,y}$ és  $R_{y,x}$ bináris kényszerek ugyanazt fejezik ki,
        ezért az irányított élek helyett irányítatlan éleket tartalmazó gráfot
        szokás használni.

        Annak, hogy $R_{x,y}$ és  $R_{y,x}$ együtt szerepel a kényszerek között,
        a később bemutatott algoritmusokban lesz jelentősége.
    \end{megjegyzes}
\end{definicio}

\begin{definicio}
    Nem bináris kényszerek.

    {\bf Unáris kényszerek:} Az unáris kényszerek egyetlen változó értékét
    korlátoznák. A helyett, hogy ezt kényszerként jelenítenénk meg,
    kifejezhetjük azzal, hogy a változóhoz rendelt tartomány elemeiként eleve
    csak az unáris kényszernek megfelelő értékeket választunk.

    {\bf Magasabb rendű kényszerek:}  A magasabb rendű kényszerek mindig
    kiválthatók bináris kényszerekkel újabb változók bevezetése mellett.
\end{definicio}

\begin{definicio}
    Kényszerek terjesztése.
    A \textbf{kényszerek terjesztése} során egy tekintett változó értékére
    vonatkozó megszorítás következményeit a vele kényszerek útján kapcsolatban
    álló változók értékeire vonatkozóan is kiterjesztjük, ezen változók
    értékeire újabb megszorításokat alkalmazva.
\end{definicio}

\begin{definicio}
    Élkonzisztencia algoritmusok.
    Az \textbf{élkonzisztencia algoritmusok} feladata, a kényszerek
    terjesztésének hatékony megvalósítása. Ezen algoritmusok egyaránt
    alkalmazhatóak a keresés megkezdése előtt a probléma méretét csökkentő
    előfeldolgozó lépésként, vagy akár a keresés közben is. Ezek közül mi az
    előbbi lehetőséget vizsgáljuk meg.
\end{definicio}

\subsection{AC1}

Az AC-1 (gyakorlatban nem használt) algoritmus egyszerű naiv megközelítés
segítségével mutatja be a kényszerek terjesztésének ötletét.

\begin{definicio}
    Felülvizsgálat.
    Az $R_{x,y}$ \textbf{kényszer felülvizsgálat}a során a
    $\langle \mathcal{V}, \mathcal{D}, \mathcal{C} \rangle$
    bináris kényszereket tartalmazó véges kényszerkielégítési problémát a
    $\langle \mathcal{V}, \mathcal{D}', \mathcal{C}' \rangle$
    problémával helyettesítjük, ahol \[
    \mathcal{D}' =
    \begin{cases}
        \{a : a \in \mathcal{D}(x) \land b \in D(y) \land \langle a, b\rangle \in R_{x,y}\}
        &\text{ ha } z=x \\
        \mathcal{D}(z)
        &\text{ egyébként.}
    \end{cases}
    .\]
\end{definicio}

\begin{algorithm}[H]
    \DontPrintSemicolon
    \SetKwFunction{FACone}{AC1}
    \SetKwProg{Fn}{Function}{:}{}
    \Fn{\FACone{$\langle \mathcal{V}, \mathcal{D}, \mathcal{C} \rangle$}}
    {
        \Repeat{$\mathcal{D} = \mathcal{D}'$}{
            $\mathcal{D}' \gets \mathcal{D}$ \;
            \ForAll{$R_{x,y} \in \mathcal{C}$}
            {
                $\mathcal{D} \gets
                \revise\left(R_{x,y}, \langle \mathcal{V}, \mathcal{D},
                \mathcal{C} \rangle  \right)$
            }
        }
        \KwRet{$\langle \mathcal{V}, \mathcal{D}, \mathcal{C} \rangle$}\;
    }
    \caption{AC-1}
\end{algorithm}

\subsection{AC3}

Az AC-3 algoritmus (Mackworth) egy már a gyakorlatban is használható
algoritmus, amely egy sorban tartja nyilván azokat a kényszereket, amelyeket
felül kell vizsgálni.

Ha $R_{x,y}$ felülvizsgálat során az  $x$ változó tartománya megváltozik, akkor
minden olyan kényszer ismét bekerül a sorba, amely $R_{y,x}$ alakú.

Az algoritmus akkor fejeződik be, mikor elfogytak a sorból a kényszerek.

Az algoritmus időbonyolultsága $\mathcal{O}(n^2 \cdot d^3)$.

\begin{algorithm}[H]
    \DontPrintSemicolon
    \SetKwFunction{FACthree}{AC3}
    \SetKwProg{Fn}{Function}{:}{}
    \Fn{\FACthree{$\langle \mathcal{V}, \mathcal{D}, \mathcal{C} \rangle$}}
    {
        \While{$W \neq \varnothing$}{
            remove an $R_{x,y}$ constraint from $W$\;
            $\mathcal{D}' \gets \mathcal{D}$ \;
            $\mathcal{D} \gets
            \revise\left(R_{x,y}, \langle \mathcal{V}, \mathcal{D},
            \mathcal{C} \rangle  \right)$\;
            \If{$\mathcal{D}' \neq \mathcal{D}$}{
                \eIf{$\mathcal{D}(x) = \varnothing$}{
                    \KwRet{failure}\;
                }{%
                    \ForAll{$R_{u,w} \in \{R_{u,w} : R_{u,w} \in
                        \mathcal{C} \land w = x\} $}{
                        $W \gets W \cup \{R_{u,w}\}$ \;
                    }
                }
            }
        }
        \KwRet{$\langle \mathcal{V}, \mathcal{D}, \mathcal{C} \rangle$}\;
    }
    \caption{AC-3}
\end{algorithm}

\subsection{AC4}

Az AC-4 algoritmus (Mohr és Henderson) két előre elkészített
adatszerkezettel dolgozik:

\begin{itemize}
    \item minden $R_{x,y}$ kényszer és $v_x \in  \mathcal{D}(x)$ értékhez egy
         számlálót rendelünk: \[
             C_{x,v_x,y} =
             \card \{v_y : v_y \in \mathcal{D}(y) \land \langle v_x, v_y \rangle \in R_{x,y}\}
         ,\]
     \item az $x$ változó minden $v_x \in \mathcal{D}(x)$ értékéhez egy halmazt
         rendelünk: \[
            S_{x,v_x} = \{
                \left<y, v_y \right> :
                v_y \in \mathcal{D}(y) \land
                \left<v_y, v_x \right> \in R_{y,x}
            \}
        ,\]
        amely megmutatja, hogy mely $y$ változók mely $v_y \in \mathcal{D}(y)$ értékeihez
        számoltuk hozzá $v_x$-et.
\end{itemize}

Jelölje $\mathcal{D}_{-\left<x, v_x \right>}$ azt a leképezést, amely
$\mathcal{D}$ leképezéstől annyiban különbözik, hogy $v_x \not\in
\mathcal{D}(x)$, vagyis $x$ nem veheti fel $v_x$ értéket: \[
    \mathcal{D}_{-\left<x, v_x \right>}(z) =
    \begin{cases}
        \mathcal{D}(x)\setminus \{v_x\} & \text{ ha } z=x,\\
        \mathcal{D}(z) & \text{ egyébként.}
    \end{cases}
\]

\begin{algorithm}[H]
    \DontPrintSemicolon
    \SetKwFunction{FACfourinit}{AC4-initialize}
    \SetKwProg{Fn}{Function}{:}{}
    \Fn{\FACfourinit{$\langle \mathcal{V}, \mathcal{D}, \mathcal{C} \rangle$}}
    {
        Compute $S$ \;
        Compute $C$ \;
        \ForAll{$\left<x, v_x \right> \in \{\left<x, v_x \right> : C_{x,v_x,y}
            = 0\}$}{
            $\mathcal{D} \gets \mathcal{D}_{-\left<x, v_x \right>}$
        }
        \KwRet{$S, C, \mathcal{D}$}\;
    }
    \caption{AC-4 inicializálás}
\end{algorithm}

\begin{algorithm}[H]
    \DontPrintSemicolon
    \SetKwFunction{FACfour}{AC4}
    \SetKwProg{Fn}{Function}{:}{}
    \Fn{\FACfour{$\langle \mathcal{V}, \mathcal{D}, \mathcal{C} \rangle$}}
    {
        $S, C, D \gets$ \FACfourinit{$\left< \mathcal{V}, \mathcal{D},
        \mathcal{C} \right>$} \;

        $W \gets \{\left<x, v_x \right> : C_{x,v_x,y} = 0\} $

        \While{$W \neq \varnothing$}{
            remove an $\left<x,v_x \right>$ pair from $W$ \;
            \ForAll{$\left<y, v_y \right> \in S_{x, v_x}$}{
                $C_{y, v_y, x} \gets C_{y, v_y, x} - 1$ \;
                \If{$C_{y, v_y, x} = 0$ \textbf{and} $v_y \in \mathcal{D}_y$}
                {
                $\mathcal{D} \gets \mathcal{D}_{-\left<y, v_y \right>}$ \;
                insert the $\left<y, v_y \right>$ pair into $W$\;
                }
            }
        }
        \KwRet{$S, C, \mathcal{D}$}\;
    }
    \caption{AC-4}
\end{algorithm}

Az {\bf AC-4 algoritmus} egy váltótó-érték párokat tartalmazó munkahalmaz
segítségével végzi el feladatát. Azon értékek, amelyek által $C$ számlálója $0$-ra
csökken törölni kell a változó tartományából. A munkahalmazba a törtölt értékek
kerülnek, mert a tőlük függő értékek számlálóit csökkenteni kell. Ezek gyors
felderítésében segít $S$.

\begin{megjegyzes}
    Az AC-4 algoritmus időbonyolultsága: $\mathcal{O}(n^2 \cdot d^2)$. Ennek ellenére valódi
    problémák esetében sok esetben az AC-3 algoritmus teljesít jobban (Richard
    J.  Wallace).
\end{megjegyzes}

\subsection{Visszalépéses kereső}


\chapter{Témakörök, melyekre rálátással kell rendelkezni}

\section{Következtetések ítéletlogikában}


\begin{definicio}
    Vonzat.
    Azt mondjuk, hogy az $\alpha$ mondat maga után vonzza a  $\beta$ mondatot,
    akkor és csakis akkor, ha minden modellben, amelyben $\alpha$ igaz, $\beta$ szintén
    igaz. Ezt a következőképp jelöljük: \[
        \alpha \models \beta
    .\]
\end{definicio}

\begin{definicio}
    Következtetési eljárás helyessége.

    Egy következtetési eljárást, amely csak vonzat mondatokat vezet le,
    helyesnek vagy igazságtartónak nevezzük. A helyesség egy igencsak kívánatos
    tulajdonság. Egy nem helyes következtetési eljárás kitalál olyan dolgokat
    ahogy előrehalad, olyan tűk megtalálását jelenti be, amelyek nem is
    léteznek.
\end{definicio}

\begin{definicio}
    Következtetési eljárás teljessége.  A teljesség tulajdonság szintén
    kívánatos: {\bf egy következtetési eljárás teljes, ha képes levezetni
    minden vonzatmondatot}.

    Valódi szénakazlak esetében, amelyek véges méretűek, nyilvánvalónak tűnik,
    hogy szisztematikus kutatással mindig eldönthető, hogy a tű a kazalban
    van-e. Sok tudásbázis esetében azonban a konzekvenciák szénakazlának mérete
    végtelen, és így a teljesség egy fontos kérdéssé válik.
\end{definicio}

\subsection{Rezolúciókalkulus ítéletlogikában}

\begin{definicio}
    Konjunktív normálforma (KNF).

    Egy mondatot, amelyet literálok diszjunkcióinak konjunkcióival fejezünk ki,
    {\bf konjunktív normál formájúnak} (conjunctive normal form) vagy KNF
    formájúnak nevezünk.

    Például: \[
        (X_1 \lor X_2 \lor \ldots X_n) \land
        (Y_1 \lor Y_2 \lor \ldots Y_n)
    .\]
\end{definicio}

\begin{definicio}
    Rezolúció.
    A rezolúció egy levezetési eljárás, mely alapja az egyik automatikus
    tételbizonyítási módszernek, elméletnek, a rezolúciós kalkulusnak.

    A rezolúció szintatikus módszer; alapja, hogy két logikai formulához
    hozzárendeljük egy speciális következményformulájukat, az ún.
    {\bf rezolvens}üket.
\end{definicio}

\begin{definicio}
    Zárt klóz.  Véges sok negálatlan vagy negált zárt atomi formulából
    (literálból) diszjunkció által összetett formula egy {\bf zárt klóz}t
    alkolt. Például \[
        X \lor Y \lor Z
    .\]

    Ennek speciális esete a {\bf Horn-klóz}, amely olyan zárt klóz, mely
    legfeljebb egy negálatlan atomot tartalmaz, a többi tagja viszont negált.
    Például \[
        A \lor \lnot B \lor \lnot C \lor \lnot D
    .\]
\end{definicio}

\begin{megjegyzes}
    A rezolúció felhasználható például SAT-problémák (kielégíthetőségi
    problémák) megoldására, azaz amikor egy formula KNF alakjáról el kell
    dönteni, hogy található-e hozzá olyan interpretáció, ahol a formula
    kielégíthető.
\end{megjegyzes}

\begin{megjegyzes}
    A rezolúció azért használatos, mert lényegesen gyorsabb, mint ha az
    ítéletlogikai szemantikai definíciók mentén számolnánk.
\end{megjegyzes}

A rezolúción alapuló következtetési eljárások az ellentmondásokra vezető
bizonyítások elvén működnek. Tehát annak megmutatásához, hogy $\text{TB}
\models \alpha$, azt mutatjuk meg, hogy a $(\text{TB} \land \lnot \alpha)$
kielégíthetetlen. Ezt az ellentmondás bizonyításával végezzük el.

\begin{algorithm}[H]
    \label{alg:rezolval}
    \caption{Egy egyszerű rezolúciós algoritmus ítéletkalkulushoz}
    \DontPrintSemicolon
    \SetKwFunction{FIDthree}{ID3}
    \SetKwProg{Fn}{Function}{:}{}
    \Fn{\Frezolval{TB, $\alpha$}}
    {
        \Input{TB, a tudásbázis, egy ítéletkalkulus mondat\\$\alpha$, a
        lekérdezés, egy ítéletkalkulus mondat}

        klózok $\gets$ a TB $\land \lnot \alpha$ mondatot reprezentáló CNF formulájú klózok halmaza \;

        új  $\gets \{ \}$ \;

        \Loop{}{
            \ForEach{$C_i, C_j$ {\bf in} klózok}{
                rezolvensek $\gets$ \Frezolval{$C_i, C_j$} \;

                \lIf{rezolvensek tartalmazzák az üres klózt}{
                    \KwRet{igaz}
                }

                új $\gets$ új  $\cup$ rezolvensek
            }
            \lIf{új $\subseteq$ klózok}{\KwRet{hamis}}
        }
    }
\end{algorithm}

A rezolúció algoritmust mutatja a \ref{alg:rezolval}. algoritmus. Először a
$(\text{TB} \land \lnot \alpha)$-t konvertáljuk CNF formára. Majd a rezolúciós
szabályt alkalmazzuk a létrejövő klózokra. Minden egyes párt, amely kiegészítő
literálokat tartalmaz, rezolválunk, hogy egy új klózt hozzunk létre, amelyet
hozzáadunk a halmazhoz, ha még nem volt jelen. A folyamat addig folytatódik,
amíg a következő két dolog közül valamelyik meg nem történik:
\begin{itemize}
    \item nincs több új klóz, amit hozzá lehet adni, ilyen esetben $\alpha$ nem
        vonzza maga után  $\beta$-t
    \item a rezolúció alkalmazása egy üres klózra vezet, amely esetben $\alpha
        \models \beta$.
\end{itemize}

\begin{tetel}
    Az üres klóz -- egy diszjunkt nélküli diszjunkció -- ekvivalens a {\it
    Hamis} értékkel, mert a diszjunkció akkor igaz csak, ha legalább az egyik
    diszjunkt igaz. Vagyis beláthatjuk, hogy az üres klóz ellentmondást
    reprezentál, hogy megfigyeljük azt, hogy az üres klóz két kiegészítő
    egységklóz, mint amilyen az $S$ és $\lnot S$, rezolválásából származik.
\end{tetel}

\section{Döntési fák}

\subsection{Az ID3 algoritmus}

\section{Valószínűségi következtetés}

\subsection{Bayes hálok, Bayes tétel}

\subsection{Feltételes valószínűség számítása}

\section{Neurális hálók, kitekintés}

\subsection{Nuerális hálók és mélytanulás}


% end chapters

\begin{thebibliography}{9}
\bibitem{nagykonyv}
\textbf{Russell} Stuart J, \textbf{Norvig} Peter:
Mesterséges Intelligencia modern megközelítésben.
Panem Kft., 2005.

\bibitem{varteresz}
Várterész Magda:
A mesterséges intelligencia alapjai: Az előadások mellé vetített anyag
(2011).
\end{thebibliography}


\end{document}
