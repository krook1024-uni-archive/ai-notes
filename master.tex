\documentclass[a4paper]{report}

\title{Mesterséges intelligencia alapjai jegyzet}
\author{Molnár Antal Albert, Lovász Botond}

\usepackage[utf8]{inputenc}
\usepackage[T1]{fontenc}
\usepackage{textcomp}
\usepackage[magyar]{babel}
\usepackage{amsmath, amsfonts, mathtools, amsthm, amssymb}
\usepackage{graphicx}
\usepackage{float}
\usepackage{titlesec}
\usepackage{algorithmicx}
\usepackage{parskip}
\usepackage{tabularx}
\usepackage{hyperref}
\usepackage[ruled,longend,linesnumbered]{algorithm2e}
\usepackage[margin=2cm]{geometry}

%\let\stdsubsection\subsection
%\renewcommand\subsection{\clearpage\stdsubsection}

%\newcommand{\chapterbreak}{\newpage}
%\newcommand{\subsectionbreak}{\newpage}

\usepackage{tikz}
\usetikzlibrary{positioning}
\tikzset{set/.style={draw,circle,inner sep=0pt,align=center}}

\usepackage{etoolbox}

%\preto{\section}{\clearpageafterfirst}
%\preto{\subsection}{\filbreak}
%\newcommand{\clearpageafterfirst}{%
%  \gdef\clearpageafterfirst{\clearpage}%
%}
%

\newtoggle{aftersection}
\preto{\section}{\filbreak\global\toggletrue{aftersection}}
\preto{\subsection}{\iftoggle{aftersection}{\global\togglefalse{aftersection}}{\filbreak}}
\newcommand{\clearpageafterfirst}{%
  \gdef\clearpageafterfirst{\clearpage}%
}

% document settings
\setlength{\parindent}{0cm}
\graphicspath{ {./figures/} }

% figure support
\usepackage{import}
\usepackage{xifthen}
\pdfminorversion=7
\usepackage{pdfpages}
\usepackage{transparent}
\newcommand{\incfig}[1]{%
    \def\svgwidth{\columnwidth}
    \import{./figures/}{#1.pdf_tex}
}

%setup alg2e
\DontPrintSemicolon
\SetKwFunction{Ftreesearch}{Fa-Kereses}
\SetKwFunction{Fgraphsearch}{Graf-Kereses}
\SetKwFunction{FACone}{AC1}
\SetKwFunction{FACthree}{AC3}
\SetKwFunction{FACfourinit}{AC4-initialize}
\SetKwFunction{FACfour}{AC4}
\SetKwFunction{Frezolval}{Rezolválás}
\SetKwFunction{FACfour}{AC4}
\SetKwFunction{Fbacktrack}{Visszalépéses-keresés}
\SetKwFunction{Fbacktrackrec}{Rekurzív-Visszalépéses}
\SetKwProg{Fn}{function}{:}{end function}
\SetKwFor{Loop}{loop}{do}{end loop}
\SetKwInOut{Input}{input}
\SetKwInOut{Output}{output}

% Environments
\makeatother
% For box around Definition, Theorem, \ldots
\usepackage{mdframed}
\mdfsetup{skipabove=1em,skipbelow=0em}
\theoremstyle{definition}
\newmdtheoremenv[nobreak=true]{definicio}{Definíció}
\newmdtheoremenv[nobreak=true]{pelda}{Példa}
\newmdtheoremenv[nobreak=true]{megjegyzes}{Megjegyzés}
\newmdtheoremenv[nobreak=true]{lemma}{Lemma}
\newmdtheoremenv[nobreak=true]{tetel}{Tétel}
\newmdtheoremenv[nobreak=true]{informacio}{Információ}
\newmdtheoremenv[nobreak=true]{konkluzio}{Konkluzio}
\newmdtheoremenv[nobreak=true]{sejtes}{Sejtés}
\newmdtheoremenv[nobreak=true]{allitas}{Állítás}
\newmdtheoremenv[nobreak=true]{tulajdonsagok}{Tulajdonságok}

\newtheorem*{ismetles}{Ismétlés}
\newtheorem*{bizonyitas}{Bizonyítás}
\newtheorem*{gyakorlas}{Gyakorlás}
\newtheorem*{problema}{Probléma}
\newtheorem*{terminologia}{Terminológia}
\newtheorem*{spec}{Speciálisan}
\makeatletter

\DeclareMathOperator{\card}{card}
\DeclareMathOperator{\dom}{dom}
\DeclareMathOperator{\revise}{revise}
\DeclareMathOperator{\rng}{rng}

\pdfsuppresswarningpagegroup=1

\begin{document}

\maketitle

\section*{Előszó}
Ez a jegyzet a Debreceni Egyetemen a Kádek Tamás által oktatott,
\textit{INBPM0418E} tárgykódú \textit{A mesterséges intelligencia alapjai}
tárgyhoz nyújt némi segítséget a vizsgára készülő hallgatóknak.

A leírtakért semmilyen felelősséget nem tudok vállalni, hiszen még jómagam is
csak ismerkedem a mesterséges intelligencia világával.

A jegyzet nagyrészt az órákon tárgyalt könyvből\cite{nagykonyv}, valamint
Várterész Magda előadásaiból\cite{varteresz} merít.


\tableofcontents

% begin chapters
\chapter{Témakörök, melyeket mélységben ismerni kell}

\section{Ágens szemlélet}

\subsection{Az ágens fogalma}

\subsection{Az ágens jellemzése (teljesítmény, környezet, érzékelők, beavatkozók}

\section{Állapottér reprezentáció}

\subsection{Az állapottér fogalma}

Az állapottér-reprezentáció az egyik leggyakrabban használt reprezentációs mód
ami egy probléma formális megadására szolgál.

\begin{definicio}
    Probléma.
    Egy \textbf{probléma} (problem) formális megragadásához az alábbi négy
    komponensre van szükség:

    \begin{itemize}
        \item \textbf{kiinduló állapot}:
            amiből az ágens kezdi a cselekvéseit,
        \item \textbf{cselekvések halmaza}:
            ágens rendelkezésére álló lehetséges cselekvések,
        \item \textbf{állapotátmenet-függvény}:
            visszaadja a rendezett $\langle$ cselekvés, utódállapot $\rangle$
            párok halmazát (Egy alternatív megfogalmazás az \textbf{operátor}ok
            egy halmaza, amelyeket egy állapotra alkalmazva lehet az
            utódállapotokat generálni.) lásd még:
            \textit{\ref{operator-alkalmazasi}. Definíció}
        \item \textbf{állapottér}: A kezdeti állapot és az
            állapotátmenet-függvény együttesen implicit módon definiálják a
            probléma \textbf{állapotteré}t: azon állapotok halmazát, amelyek a
            kiinduló állapotból elérhetők.
    \end{itemize}
\end{definicio}

\begin{definicio}\label{operator-alkalmazasi}
    Operátor alkalmazási előfeltétel teszt.
    Ahhoz, hogy meggyőződjünk arról, hogy egy adott állapotra egy adott
    operátor alkalmazható, előbb meg kell vizsgálnunk, hogy az állapotra
    alkalmazható-e az operátor. Ezt a vizsgálódást operátor alkalmazási
    előfeltétel tesztnek nevezzük.
\end{definicio}

\begin{definicio}
    Célteszt.
    A célteszt ellenőrzi a célfeltételek teljesülését egy adott állapotban.
\end{definicio}

\subsection{Az állapottérgráf}

\begin{definicio}
    Állapottérgráf.
    Az állapottér egy gráfot alkot,
    amelynek csomópontjai az állapotok és a csomópontok közötti élek a
    cselekvések.


\end{definicio}

\begin{definicio}
    Állapottér útja.
    Az \textbf{állapottér egy útja} az állapotok egy sorozata, amely
    állapotokat a cselekvések egy sorozata köt össze.
\end{definicio}

\begin{definicio}
    Állapottér-reprezentációs gráf bonyolultsága.  Egy állapottér-reprezentált
    probléma megoldásának sikerét jelentősen befolyásolja a reprezentációs gráf
    bonyolultsága:

    \begin{itemize}
        \item a csúcsok száma,
        \item az egy csúcsból kiinduló élek száma,
        \item a hurkok és körök száma és hossza.
    \end{itemize}

    Ezért célszerű minden lehetséges egyszerűsítést végrehajtani. Lehetséges
    egyszerűsítések:

    \begin{itemize}
        \item a csúcsok számának csökkentése — ügyes reprezentációval az
            állapottér kisebb méretű lehet;
        \item az egy csúcsból kiinduló élek számának csökkentése — az
            operátorok értelmezési tartományának alkalmas megválasztásával
            érhető el;
        \item a reprezentációs gráf fává alakítása — a hurkokat, illetve
            köröket „kiegyenesítjük”
    \end{itemize}
\end{definicio}

\subsubsection{Az állapottérgráf jellemzése}

$b$: Az {\bf elágazási tényező} (branching factor) a tetszőleges
állapotból közvetlenül elérhető állapotok maximális száma. \[
\max \{\card \{b:a \Rightarrow b\} : a \in \mathcal{A})\}
.\]

$d$ : A {\bf legsekélyebb megoldás} a legkevesebb operátoralkalmazás
segítségével elérhető célállapot eléréshez szükséges operátoralkalmazások száma.
(A legrövidebb megoldás hossza. A célállapotok minimális mélysége.)
A legkisebb olyan $i$ amely esetén van olyan állapotsorozat, hogy \[
    a_0 \Rightarrow a_1, \quad a_1 \Rightarrow a_2 \quad \ldots \quad a_{i-1}
    \Rightarrow a_i,\quad a_i \in \mathcal{C}
.\]

$m$ : A {\bf csomópontok maximális mélysége}. A legnagyobb olyan $i$, amely esetén
van olyan állapotsorozat, hogy \[
    a_0 \Rightarrow a_1,\quad a_1 \Rightarrow a_2 \quad \ldots \quad a_{i-1}
    \Rightarrow a_i
.\]

% TODO

\subsection{Költség és heurisztika fogalmak}

\begin{definicio}
    Lépésköltség.
    Az $x$ állapotból az $y$ állapotba vezető $cs$ cselekvés
    \textbf{lépésköltség}e (step cost) legyen $lk(x, cs, y)$.  Tételezzük fel,
    hogy a lépésköltség nemnegatív.
\end{definicio}

\begin{definicio}
    Útköltség-függvény.
    Egy \textbf{útköltség-függvény}, egy olyan függvény amely az állapottér
    minden útjához hozzárendel egy költséget.
\end{definicio}

\begin{definicio}
    Megoldás.
    Egy út, amely a kiinduló állapotból egy célállapotba vezet.
\end{definicio}

\begin{definicio}
    Optimális megoldás.
    A legkisebb útköltségű megoldás.
\end{definicio}

% TODO

\section{Megoldáskereső algoritmusok}

\subsection{Fakereső algoritmusok}

Az általános fakeresési algoritmus informális leírása:

\begin{algorithm}[H]
    \Fn{\Ftreesearch{probléma, stratégia}}
    {
        a probléma kezdeti állapotából kiindulva inicializáld
        a keresési fát \;
        \Loop{}{
            \If{nincs kifejtendő csomópont}{\KwRet{kudarc}}

            a stratégiának megfelelően válassz ki kifejtésre egy
            levélcsomópontot

            \eIf{a csomópont célállapotot tartalmaz}{
                \KwRet{a hozzá tartozó megoldás}
            }{
                fejtsd ki a csomópontot és az eredményül kapott csomópontokat, %
                és add a keresési fához \;
            }
        }
    }
    \caption{Általános fakeresési algoritmus informális leírása}
\end{algorithm}

\begin{definicio}
    Perem.

    A legenerált, kifejtésre váró csomópontokat külön nyilvántartjuk, ez a {\bf
    perem}. Ezt a gyakorlatban általában egy várakozási sorként (queue) szokás
    implementálni.
\end{definicio}

\begin{definicio}
    Az algoritmusok hatékonyságának értékelése.

    A problémamegoldó algoritmus kimenete vagy kudarc, vagy egy megoldás (egyes
    algoritmusok végtelen hurokba kerülhetnek és soha nem térnek vissza válasszal).
    Mi az algoritmusok hatékonyságát négyféle módon fogjuk értékelni:
    \begin{itemize}
        \item {\bf Teljesség (completeness)}: az algoritmus generáltan megtalál
            egy megoldást, amennyiben létezik megoldás?
        \item {\bf Optimalitást (optimality)}: a stratégia megtalálja az optimális megoldást?
        \item {\bf Időigény (time complexity)}: mennyi ideig tart egy megoldás megtalálása?
        \item {\bf Tárigény (space complexity)}: a keresés elvégzéséhez mennyi
            memóriára van szükség?
    \end{itemize}
\end{definicio}

\subsection{Gráfkereső algoritmusok}

% TODO

\subsection{Szélességi kereső}

A {\bf szélességi keresés (breadth-first search)} egy egyszerű keresési
stratégia, ahol először a gyökércsomópontot fejtjük ki, majd a következő
lépésben az összes gyökércsomópontból generált csomópontot, majd azok követőit,
stb.

Általánosságban {\bf a keresési stratégia minden adott mélységű csomópontot hamarabb
fejt ki, mielőtt bármelyik, egy szinttel lejjebbi csomópontot kifejtené}.

A szélességi keresést meg lehet valósítani a FA-KERESÉS algoritmussal egy olyan
üres peremmel, amely egy először-be-először-ki (first-in-first-out – FIFO) sor,
biztosítva ezzel, hogy a korábban meglátogatott csomópontokat az algoritmus
korábban fejti ki.

\begin{figure}[H]
    \centering
    \includegraphics[width=0.8\textwidth]{bfs}
    \caption{Szélességi keresés egy egyszerű bináris fában}
    \label{fig:bfs}
\end{figure}

\subsubsection{Jellemzés}

\begin{itemize}
    \item {\bf Időbonyolultság}: $1 + b + b^2 + \ldots + b^{d+1} - b \in O(b^{d+1})$
    \item {\bf Tárbonyolultság}: $1 + b + b^2 + \ldots + b^{d+1} - b \in O(b^{d+1})$
    \item {\bf Teljesség}: teljes, ha $b$ véges,
    \item {\bf Optimalitás}: optimális, ha minden költség egységnyi.
\end{itemize}

\subsection{Mélységi kereső}

A {\bf mélységi keresés (depth-first search)} mindig a keresési fa aktuális
peremében a legmélyebben fekvő csomópontot fejti ki. A keresés azonnal a fa
legmélyebb szintjére jut el, ahol a csomópontoknak már nincsenek követőik.
Kifejtésüket követően kikerülnek a peremből és a keresés "visszalép" ahhoz a
következő legmélyebben fekvő csomóponthoz, amelynek vannak még ki nem fejtett
követői.

Ez a stratégia egy olyan FA-KERESÉS függvénnyel implementálható, amelynek a
sorbaállító függvénye az {\it utolsónak-be-elsőnek-ki (last-in-first-out,
LIFO)}, más néven verem. A mélységi keresést szokás a FA-KERESÉS függvény
alternatívájaként egy rekurzív függvénnyel is implementálni, amely a
gyermekcsomópontokkal meghívja önmagát.

A mélységi keresés nagyon szerény tárigényű. Csak egyetlen, a
gyökércsomóponttól egy levélcsomópontig vezető utat kell tárolnia, kiegészítve
az út minden egyes csomópontja melletti kifejtetlen csomópontokkal. Egy
kifejtett csomópont el is hagyható a memóriából, feltéve, hogy az összes
leszármazottja meg lett vizsgálva. Egy $b$ elágazási tényezőjű és m maximális
mélységű állapottér esetén a mélységi keresés tárigénye $b\cdot m + 1$.

\subsubsection{Jellemzés (fakeresőként)}

\begin{itemize}
    \item {\bf Időbonyolultság}: $1 + b + b^2 + \ldots + b^m \in O(b^m)$
    \item {\bf Tárbonyolultság}: $1 + b\cdot m \in O(b \cdot m)$, feltéve, hogy
        minden olyan csomópontot elhagyunk, amely összes leszármazottja meg
        lett vizsgálva
    \item {\bf Teljesség}: csak véges körmentes gráfban teljes
    \item {\bf Optimalitás}: nem garantál optimális megoldásokat
\end{itemize}

\subsection{Visszalépéses kereső}

A mélységi keresés {\bf visszalépéses keresésnek (backtracking search)} nevezett
változata még kevesebb memóriát használ. A visszalépéses keresés az összes
követő helyett egyidejűleg csak egy követőt generál.  Minden részben kifejtett
csomópont emlékszik, melyik követője jön a legközelebb. Ily módon csak $O(m)$
memóriára van szükség, $O(b\cdot m)$ helyett. A visszalépéses keresés még egy memória-
(és idő-) spóroló trükkhöz folyamodik. Az ötlet a követő csomópont generálása
az aktuális állapot módosításával, anélkül hogy az állapotot átmásolnánk. Ezzel
a memóriaszükséglet egy állapotra és $O(m)$ cselekvésre redukálódik. Ahhoz, hogy
az ötlet működjön, amikor visszalépünk, hogy a következő követőt generáljuk,
mindegyik módosítást vissza kell tudnunk csinálni. Nagy állapottérrel
rendelkező problémák esetén, mint például robot-összeszerelés esetén, az ilyen
módszerek lényegesek a sikerességhez.

A mélységi keresés hátrányos tulajdonsága, hogy egy rossz választással egy
hosszú (akár végtelen) út mentén lefelé elakadhat, miközben például egy más
döntés elvezetne a gyökérhez közeli megoldáshoz. A legrosszabb esetben a
mélységi keresés a keresési fában az összes $O(b^m)$ csomópontot generálni fogja,
ahol $m$ a csomópontok maximális mélysége. Jegyezzük meg, hogy $m$ sokkal nagyobb
lehet, mint $d$ (a legsekélyebb megoldás mélysége), és korlátlan fák esetén
értéke végtelen.

\begin{megjegyzes}
    A visszalépéses kereső használatával a mélységi keresés tárbonyolultsága
    csökkenthető tovább.
\end{megjegyzes}

\subsubsection{Jellemzés}

\begin{itemize}
    \item {\bf Időbonyolultság}: $1 + b + b^2 + \ldots + b^m \in O(b^m)$
    \item {\bf Tárbonyolultság}: $1 + m \in O(m)$
    \item {\bf Teljesség}: csak véges körmentes gráfban teljes
    \item {\bf Optimalitás}: nem garantál optimális megoldásokat
\end{itemize}

\subsection{Egyenletes költségű (optimális kereső)}

A szélességi keresés optimális, ha minden lépés költsége azonos, mert mindig a
legsekélyebb ki nem fejtett csomópontot fejti ki. Egyszerű általánosítással egy
olyan algoritmust találhatunk ki, amely tetszőleges lépésköltség mellett
optimális. Az {\bf egyenletes költségű keresés (uniform cost search)} mindig a
legkisebb útköltségű $n$ csomópontot fejti ki először, nem pedig a legkisebb
mélységű csomópontot. Egyszerűen belátható, hogy a szélességi keresés is
egyenletes költségű keresés, amennyiben minden lépésköltség azonos.

Az egyenletes költségű keresés nem foglalkozik azzal, hogy {\it hány} lépésből
áll egy bizonyos út, hanem csak az összköltségükkel törődik. Emiatt mindig
végtelen hurokba kerül, ha egy csomópont kifejtése zérus költségű cselekvéshez
és ugyanahhoz az állapothoz való visszatérést eredményez (például a NOOP
cselekvés).

A teljességet csak úgy garantálhatjuk, hogy minden lépés költsége
egy kis pozitív e konstansnál nagyobb, vagy azzal egyenlő. Ez a feltétel egyben
az optimalitás elégséges feltétele is. Ez azt jelenti, hogy egy út költsége az
út mentén mindig növekszik. Ebből a tulajdonságból látszik, hogy az algoritmus
a csomópontokat mindig a növekvő útköltség függvényében fejti ki. Azaz az első
kifejtésre kiválasztott célcsomópont egyben az optimális megoldás is
(emlékezzünk arra, hogy a FA-KERESÉS a célállapottesztet csak a kifejtésre
megválasztott csomópontokra alkalmazza).

\subsection{Legjobbat először kereső}

Informált keresési módszer.

A {\bf mohó legjobbat-először keresés (greedy best-first search)} azt a
csomópontot fejti ki a következő lépésben, amelyiknek az állapotát a
legközelebbinek ítéli a célállapothoz, abból kiindulva, hogy így gyorsan
megtalálja a megoldást. A csomópontokat az algoritmus tehát az $f(n) = h(n)$
heurisztikus függvénnyel értékeli ki.

\subsubsection{Jellemzés}

\begin{itemize}
    \item {\bf Időbonyolultság}: $O(b^m)$
    \item {\bf Tárbonyolultság}: $O(b^m)$
    \item {\bf Teljesség}: nem teljes
    \item {\bf Optimalitás}: nem optimális
\end{itemize}

\begin{megjegyzes}
    Az idő- és tárbonyolultság nagyban függ a heurisztikus függvény
    minőségétől.
\end{megjegyzes}

\subsection{Az A* algoritmus}

Informált keresési módszer.

A legjobbat-először keresés leginkább ismert változata az A* keresés. A
csomópontokat úgy értékeli ki, hogy összekombinálja $g(n)$ értékét – az aktuális
csomópontig megtett út költsége – és $h(n)$ értékét – vagyis az adott
csomóponttól a célhoz vezető út költségének becslőjét: \[
    f(n) = g(n) + h(n)
.\]

Mivel $g(n)$ megadja a kiinduló csomóponttól az n csomópontig számított
útköltséget, és h(n) az n csomóponttól a célcsomópontba vezető legolcsóbb
költségű út költségének becslője, így az alábbi összefüggést kapjuk:
\[
    f(n) = \text{a legolcsóbb, az n csomóponton keresztül vezető megoldás
    becsült költsége}
.\]

Így amennyiben a legolcsóbb megoldást keressük, ésszerű először a legkisebb
$g(n) + h(n)$ értékkel rendelkező csomópontot kifejteni. Ezen stratégia
kellemes tulajdonsága, hogy ez a stratégia több mint ésszerű: amennyiben a $h$
függvény eleget tesz bizonyos feltételeknek, az A* keresés teljes és optimális.

\subsubsection{Jellemzés}

\begin{itemize}
    \item {\bf Időbonyolultság}: $O(b^m)$
    \item {\bf Tárbonyolultság}: $O(b^m)$
    \item {\bf Teljesség}: teljes
    \item {\bf Optimalitás}: optimális, ha $h(n)$ konzisztens (vagyis monoton),
        valamint elfogadható heurisztika, azaz soha nem becsül felül (ez
        következik a konzisztenciából)
\end{itemize}

\section{Kétszemélyes játékok}

\subsection{A játékok reprezentációja}

\subsection{A játékfa}

\subsection{Nyerő stratégia}

\section{Lépésajánló algoritmusok}

\subsection{MinMax módszer}

\subsection{NegaMax módszer}

\subsection{Alfa-béta nyesés}

\section{Élkonzisztencia algoritmusok}

\subsection{AC1}

\subsection{AC3}

\subsection{AC4}

\subsection{Visszalépéses kereső}


\chapter{Témakörök, melyekre rálátással kell rendelkezni}

\section{Következtetések ítéletlogikában}

\subsection{Rezolúciókalkulus ítéletlogikában}

\section{Döntési fák}

\subsection{Az ID3 algoritmus}

\section{Valószínűségi következtetés}

\subsection{Bayes hálok, Bayes tétel}

\subsection{Feltételes valószínűség számítása}

\section{Neurális hálók, kitekintés}

\subsection{Neurális hálók és mélytanulás}


% end chapters

\begin{thebibliography}{9}
\bibitem{nagykonyv}
\textbf{Russell} Stuart J, \textbf{Norvig} Peter:
Mesterséges Intelligencia modern megközelítésben.
Panem Kft., 2005.
\end{thebibliography}


\end{document}
