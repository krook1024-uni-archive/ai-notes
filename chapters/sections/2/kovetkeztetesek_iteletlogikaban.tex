\section{Következtetések ítéletlogikában}


\begin{definicio}
    Vonzat.
    Azt mondjuk, hogy az $\alpha$ mondat maga után vonzza a  $\beta$ mondatot,
    akkor és csakis akkor, ha minden modellben, amelyben $\alpha$ igaz, $\beta$ szintén
    igaz. Ezt a következőképp jelöljük: \[
        \alpha \models \beta
    .\]
\end{definicio}

\begin{definicio}
    Következtetési eljárás helyessége.

    Egy következtetési eljárást, amely csak vonzat mondatokat vezet le,
    helyesnek vagy igazságtartónak nevezzük. A helyesség egy igencsak kívánatos
    tulajdonság. Egy nem helyes következtetési eljárás kitalál olyan dolgokat
    ahogy előrehalad, olyan tűk megtalálását jelenti be, amelyek nem is
    léteznek.
\end{definicio}

\begin{definicio}
    Következtetési eljárás teljessége.  A teljesség tulajdonság szintén
    kívánatos: {\bf egy következtetési eljárás teljes, ha képes levezetni
    minden vonzatmondatot}.

    Valódi szénakazlak esetében, amelyek véges méretűek, nyilvánvalónak tűnik,
    hogy szisztematikus kutatással mindig eldönthető, hogy a tű a kazalban
    van-e. Sok tudásbázis esetében azonban a konzekvenciák szénakazlának mérete
    végtelen, és így a teljesség egy fontos kérdéssé válik.
\end{definicio}

\subsection{Rezolúciókalkulus ítéletlogikában}
