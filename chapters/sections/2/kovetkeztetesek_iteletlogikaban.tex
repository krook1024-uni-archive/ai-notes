\section{Következtetések ítéletlogikában}


\begin{definicio}
    Vonzat.
    Azt mondjuk, hogy az $\alpha$ mondat maga után vonzza a  $\beta$ mondatot,
    akkor és csakis akkor, ha minden modellben, amelyben $\alpha$ igaz, $\beta$ szintén
    igaz. Ezt a következőképp jelöljük: \[
        \alpha \models \beta
    .\]
\end{definicio}

\begin{definicio}
    Következtetési eljárás helyessége.

    Egy következtetési eljárást, amely csak vonzat mondatokat vezet le,
    helyesnek vagy igazságtartónak nevezzük. A helyesség egy igencsak kívánatos
    tulajdonság. Egy nem helyes következtetési eljárás kitalál olyan dolgokat
    ahogy előrehalad, olyan tűk megtalálását jelenti be, amelyek nem is
    léteznek.
\end{definicio}

\begin{definicio}
    Következtetési eljárás teljessége.  A teljesség tulajdonság szintén
    kívánatos: {\bf egy következtetési eljárás teljes, ha képes levezetni
    minden vonzatmondatot}.

    Valódi szénakazlak esetében, amelyek véges méretűek, nyilvánvalónak tűnik,
    hogy szisztematikus kutatással mindig eldönthető, hogy a tű a kazalban
    van-e. Sok tudásbázis esetében azonban a konzekvenciák szénakazlának mérete
    végtelen, és így a teljesség egy fontos kérdéssé válik.
\end{definicio}

\subsection{Rezolúciókalkulus ítéletlogikában}

\begin{definicio}
    Konjunktív normálforma (KNF).

    Egy mondatot, amelyet literálok diszjunkcióinak konjunkcióival fejezünk ki,
    {\bf konjunktív normál formájúnak} (conjunctive normal form) vagy KNF
    formájúnak nevezünk.

    Például: \[
        (X_1 \lor X_2 \lor \ldots X_n) \land
        (Y_1 \lor Y_2 \lor \ldots Y_n)
    .\]
\end{definicio}

\begin{definicio}
    Rezolúció.
    A rezolúció egy levezetési eljárás, mely alapja az egyik automatikus
    tételbizonyítási módszernek, elméletnek, a rezolúciós kalkulusnak.

    A rezolúció szintatikus módszer; alapja, hogy két logikai formulához
    hozzárendeljük egy speciális következményformulájukat, az ún.
    {\bf rezolvens}üket.
\end{definicio}

\begin{definicio}
    Zárt klóz.  Véges sok negálatlan vagy negált zárt atomi formulából
    (literálból) diszjunkció által összetett formula egy {\bf zárt klóz}t
    alkolt. Például \[
        X \lor Y \lor Z
    .\]

    Ennek speciális esete a {\bf Horn-klóz}, amely olyan zárt klóz, mely
    legfeljebb egy negálatlan atomot tartalmaz, a többi tagja viszont negált.
    Például \[
        A \lor \lnot B \lor \lnot C \lor \lnot D
    .\]
\end{definicio}

\begin{megjegyzes}
    A rezolúció felhasználható például SAT-problémák (kielégíthetőségi
    problémák) megoldására, azaz amikor egy formula KNF alakjáról el kell
    dönteni, hogy található-e hozzá olyan interpretáció, ahol a formula
    kielégíthető.
\end{megjegyzes}

\begin{megjegyzes}
    A rezolúció azért használatos, mert lényegesen gyorsabb, mint ha az
    ítéletlogikai szemantikai definíciók mentén számolnánk.
\end{megjegyzes}

A rezolúción alapuló következtetési eljárások az ellentmondásokra vezető
bizonyítások elvén működnek. Tehát annak megmutatásához, hogy $\text{TB}
\models \alpha$, azt mutatjuk meg, hogy a $(\text{TB} \land \lnot \alpha)$
kielégíthetetlen. Ezt az ellentmondás bizonyításával végezzük el.

\begin{algorithm}[H]
    \label{alg:rezolval}
    \caption{Egy egyszerű rezolúciós algoritmus ítéletkalkulushoz}
    \DontPrintSemicolon
    \SetKwFunction{FIDthree}{ID3}
    \SetKwProg{Fn}{Function}{:}{}
    \Fn{\Frezolval{TB, $\alpha$}}
    {
        \Input{TB, a tudásbázis, egy ítéletkalkulus mondat\\$\alpha$, a
        lekérdezés, egy ítéletkalkulus mondat}

        klózok $\gets$ a TB $\land \lnot \alpha$ mondatot reprezentáló CNF formulájú klózok halmaza \;

        új  $\gets \{ \}$ \;

        \Loop{}{
            \ForEach{$C_i, C_j$ {\bf in} klózok}{
                rezolvensek $\gets$ \Frezolval{$C_i, C_j$} \;

                \lIf{rezolvensek tartalmazzák az üres klózt}{
                    \KwRet{igaz}
                }

                új $\gets$ új  $\cup$ rezolvensek
            }
            \lIf{új $\subseteq$ klózok}{\KwRet{hamis}}
        }
    }
\end{algorithm}

A rezolúció algoritmust mutatja a \ref{alg:rezolval}. algoritmus. Először a
$(\text{TB} \land \lnot \alpha)$-t konvertáljuk CNF formára. Majd a rezolúciós
szabályt alkalmazzuk a létrejövő klózokra. Minden egyes párt, amely kiegészítő
literálokat tartalmaz, rezolválunk, hogy egy új klózt hozzunk létre, amelyet
hozzáadunk a halmazhoz, ha még nem volt jelen. A folyamat addig folytatódik,
amíg a következő két dolog közül valamelyik meg nem történik:
\begin{itemize}
    \item nincs több új klóz, amit hozzá lehet adni, ilyen esetben $\alpha$ nem
        vonzza maga után  $\beta$-t
    \item a rezolúció alkalmazása egy üres klózra vezet, amely esetben $\alpha
        \models \beta$.
\end{itemize}

\begin{tetel}
    Az üres klóz -- egy diszjunkt nélküli diszjunkció -- ekvivalens a {\it
    Hamis} értékkel, mert a diszjunkció akkor igaz csak, ha legalább az egyik
    diszjunkt igaz. Vagyis beláthatjuk, hogy az üres klóz ellentmondást
    reprezentál, hogy megfigyeljük azt, hogy az üres klóz két kiegészítő
    egységklóz, mint amilyen az $S$ és $\lnot S$, rezolválásából származik.
\end{tetel}
