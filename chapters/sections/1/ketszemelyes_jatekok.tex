\section{Kétszemélyes játékok}

\subsection{A játékok reprezentációja}

A továbbiakban a
\begin{itemize}
    \item teljesen megfigyelhető,
    \item véges,
    \item determinisztikus,
    \item kétszemélyes,
    \item zérusösszegű
\end{itemize}
játékokkal foglalkozunk.

\begin{definicio}
    Játék reprezentációja.
    Egy játék reprezentációja megadható a
    \[
        \langle
        \mathcal{B},
        b_0,
        \mathcal{J},
        \mathcal{V},
        \hat{v},
        \mathcal{L}
        \rangle
    \]
    rendezett hatossal, ahol:
    \begin{itemize}
        \item $\mathcal{B}$: a játékállások halmaza,
        \item $b_0$: a kezdőállás, ahol $b_0 \in \mathcal{B}$,
        \item $\mathcal{J}$: a játékosok halmaza, ahol $\card \mathcal{J} = 2$,
        \item $\mathcal{V}$: a végállások halmaza, ahol $\mathcal{V} \subseteq
            \mathcal{B}$,
        \item $\hat{v}$: egy $\mathcal{V} \to \{-1, 0, 1\} $
            függvény,
        \item $\mathcal{L}$: a lépések halmaza.
    \end{itemize}
\end{definicio}

\begin{definicio}
    Nyertes.
    Végállásban a $\hat{v}$ függvény határozza meg, hogy melyik
    játékos nyert:
     \[
        \hat{v}(b) =
        \begin{cases}
            1 & \text{ha $b$ állásban a következő játékos nyer} \\
            0 & \text{ha $b$ állásban a következő játékos veszít} \\
            -1 & \text{egyébként (döntetlen esetén)}
        \end{cases}
    .\]
\end{definicio}

\begin{definicio}
    Lépés.
    Minden $l \in \mathcal{L}$ egy $\mathcal{B} \to \mathcal{B}$
    parciális függvény.

\end{definicio}

\begin{definicio}
    Játék állapottere.
    Legyen $\langle \mathcal{B}, b_0, \mathcal{J}, \mathcal{V}, \hat{v},
    \mathcal{L} \rangle$ egy játék reprezentációja. Ekkor a \textbf{játék
    állapottere} $\langle \mathcal{A}, a_0, \mathcal{C}, \mathcal{O} \rangle$
    definiálható a következőképpen:
     \begin{itemize}
         \item $\mathcal{A} = \mathcal{B} \times \mathcal{J}$;
         \item $a_0 = \langle b_0, j_0 \rangle$, ahol $j_0 \in \mathcal{J}$ a kezdőjátékos;
         \item $\mathcal{C} = \{\langle b, j \rangle : \langle b, j \rangle \in A \land b \in \mathcal{V}\} $
         \item $\mathcal{O} = \{o_l : l \in \mathcal{L}\}$, ahol $o_l :
             \mathcal{A} \to \mathcal{A}$, úgy hogy
             \begin{itemize}
                 \item $\dom(o_l) = \{ \langle b, j \rangle : b \in \dom(l)\}$
                 \item $o_l(\langle b, j \rangle) = \langle l(b), j'\rangle$
                 \item  $j' \in \left(\mathcal{J} \setminus  \{j\}
                     \right) $
             \end{itemize}

     \end{itemize}
\end{definicio}

\begin{definicio}
    Közvetlen elérhetőség.
    Az $a \in \mathcal{A}$ állapotból az $a' \in \mathcal{A}$ állapot
    \textbf{közvetlenül elérhető}, ha van olyan $o_l \in \mathcal{O}$,
    amely esetén \[
        a \in \dom(o_l) \text{  és  } o_l(a) = a'
    \]
    és ezt $a \Rightarrow a'$ alakban jelöljük.
\end{definicio}

\begin{definicio}
    Elérhetőség.
    $Az a \in \mathcal{A}$ állapot az $a' \in \mathcal{A}$ állapot
    \textbf{elérhető} ($a \Rightarrow^* a'$), ha
    \begin{itemize}
        \item $a=a'$, vagy
        \item van olyan  $a_1, a_2, \ldots, a_k$ állapotsorozat, hogy
            $a_1=a, a_k=a'$ továbbá \[
                a_i \Rightarrow a_{i+1} \text{ minden }
                i \in \{1, \ldots, k-1\} \text{ esetén}
            .\]
    \end{itemize}
\end{definicio}

\subsection{A játékfa}

\begin{definicio}
    Játékfa.
    Legyen $\langle \mathcal{B}, b_0, \mathcal{J}, \mathcal{V}, \hat{v},
    \mathcal{L} \rangle$ egy játék reprezentációja és $j_0 \in \mathcal{J}$ a
    kezdőjátékos. Ekkor a játékfa olyan fa, melynek csúcsaihoz a játék
    állapotait rendeljük:
    \begin{itemize}
        \item a fa gyökere a $\langle a_0, j_0 \rangle$ állapottal címkézett csúcs,
        \item a fa levélelemei olyan $\langle b, j \rangle$ állapottal
            címkézett csúcsok, ahol  $b \in \mathcal{V}$
        \item a fa $\langle b, j \rangle$ állapottal címkézett nem levélcsúcsának
            gyermekeit olyan $\langle b', j' \rangle$ címkéjű csúcsok alkotják, ahol
            $\langle b, j \rangle \Rightarrow \langle b', j' \rangle$
    \end{itemize}
\end{definicio}

\megjegyzes{A játékfa a játék állapotterének gráfját fává egyenesíti ki,
egy-egy állapot a fában több csúcs címkéjeként is szerepelhet.}

\subsection{Nyerő stratégia}

\begin{definicio}
    Játszma.
    Egy {\bf játszma} egy olyan lépéssorozat, amely a $\left<b_0, j_0 \right>$
    kezdőállapotból valamely $\left<b,j \right> \in \mathcal{C}$ célállapotba
    vezet. (Egy a gyökértől valamely levélcsúcsig vezető út a játékfában).
\end{definicio}

\begin{definicio}
    Stratégia.
    Játékterv, ami minden olyan állásban, amikor ő következik, megmondja a
    játékosnak, hogy mit lépjen.

    A $j\in\mathcal{J}$ játékos {\bf stratégiája} olyan $S_j : \{\left<b,j
    \right> : \left<b,j \right> \in \mathcal{A}\} \to \mathcal{O}$ leképezés
    (döntési terv), amely $j$ számára előírja, hogy a játék azon állapotaiban,
    ahol  $j$ a lépni következő játékos, a megtehető lépések közül melyiket
    lépje meg.
\end{definicio}

\begin{definicio}
    Nyerő stratégia.

    A $j \in \mathcal{J}$ játékos egy $S_j$ stratégiája {\bf nyerő stratégia},
    ha minden a stratégia mellett lejátszható játszmában $j$ nyer.

    Ha döntetlen állhat elő -- $0 \not\in \rng\left( \hat{v} \right)$ --,
    akkor az egyik játékos rendelkezik nyerő stratégiával. A nyerő
    stratégiával rendelkező játékos $w(a_0, j_0)$: \[
        w(b,j) = \begin{cases}
            j & \text{ ha } b \in \mathcal{V} \text{ és } \hat{v}(b) = 1 \\
            j' & \text{ ha } b \in \mathcal{V} \text{ és } \hat{v}(b) = -1 \\
            j & \text{ ha van olyan} \left<b', j' \right> \text{ hogy }
            \left<b, j \right> \rightarrow \left<b', j' \right> és
            w(b', j') = j\\
            j' & \text{ egyébként,}
        \end{cases}
    \]
    ahol $j' \in (\mathcal{J} \setminus \{j\})$.
\end{definicio}

\begin{megjegyzes}
    Ha a játékban van döntetlen végállás, akkor az egyik játékosnak van
    garantáltan nem vesztő stratégiája.
\end{megjegyzes}
