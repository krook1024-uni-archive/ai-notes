\section{Lépésajánló algoritmusok}

\begin{definicio}
    Hasznosságfügvény.

    A {\bf hasznosságfügvény} egy becslés, de elávrjuk, hogy végállásban pontos
    legyen. Formálisan: \[
        h  : \mathcal{A} \to [-m, m] \text{ ahol } m \in \mathbb{R}^+
        \text{ továbbá ha } b \in \mathcal{V} \text{ akkor }
        h(b,j) = m \cdot \hat{v}(b)
    .\]

    Ha $h(b,j) > h(b', j)$ akkor a $b$ állás a $j$ (lépni következő) játékos
    számára kedvezőbb, mint a $b'$ állás.

    Legyen $h$ egy hasznosságfüggvény, ekkor $h_i$ az $i \in \mathcal{J}$ játékos
    hasznosságfüggvénye, ha \[
        h_i(a,j) = \begin{cases}
            h(a,j) & \text{ ha } i = j \\
            -h(a,j) & \text{ egyébként.}
        \end{cases}
    \]
\end{definicio}

\subsection{MinMax módszer}

Az $b \in \mathcal{B}$ állás hasznossága $t \in \mathcal{J}$ támogatott játékos
számára ha $j \in \mathcal{J}$ játékos következik lépni, $k \in \mathbb{N}$
mélységi korláttal rekurzívan a következőképpen számítható:
\[
    f(b, j, t, k) =
    \begin{cases}
        h_t(b,j) & \text{ ha } b \in \mathcal{V} ,\\
        h_t(b,j) & \text{ ha } k = 0 ,\\
        \min \{ f(b', j', t, k-1) : \left<b, j \right> \Rightarrow \left<b', j' \right>\}
        & \text{ ha } t \neq j ,\\
        \max \{ f(b', j', t, k-1) : \left<b, j \right> \Rightarrow \left<b', j' \right>\}
        & \text{ ha } t = j.
    \end{cases}
\]

A MinMax algoritmus az optimális döntést az aktuális állapotból számítja ki,
felhasználva az egyes követő állapotok minimax értékeinek kiszámítására a
definiáló egyenletekből közvetlenül származtatott, egyszerű rekurzív formulát.
A rekurzió egészen a falevelekig folytatódik, majd a minimax értékeket a fa
mentén visszafelé terjesztjük, ahogy a rekurzió visszalép.  Bizonyos esetben
például az algoritmus először rekurzív módon leereszkedik a három bal alsó
csomóponthoz, a HASZNOSSÁG függvénnyel kiszámítva.  Majd az algoritmus előveszi
ezen értékek minimumát és ezt adja vissza, ahogy a szülő csomóponthoz
visszatér.

Minden olyan ponton, ahol a támogatott játékos van döntési helyzetben, akkor ő maga
számára a lehető legjobbat akarja, ezért a közvetlenül elérhető játé

Az ellenfele önmagának akarja a legjobbat, viszont ami az ellenfélnek a legjobb, az a
támogatott játékos számára a legrosszabb, tehát ezért van ebben az esetben minimum
keresés ebben az implementációban.

\subsection{NegaMax módszer}

Az $b \in \mathcal{B}$ állás hasznossága $j \in \mathcal{J}$ (lépni következő)
játékos számára $k \in \mathbb{N}$ mélységi korláttal rekurzívan a következőképpen
számítható: \[
    \hat{f} (b, j, k) = \begin{cases}
        h(b,j) & \text{ ha } b \in \mathcal{V},\\
        h(b,j) & \text{ ha } k = 0,\\
        max \{
            -\hat{f}(b', j', k-1) : \left<b,j \right> \Rightarrow \left<b',j' \right>
        \}
               & \text{ egyébként.}
    \end{cases}
\]

A NegaMax algoritmus esetében is hasonlóan kell gondolkodnunk, mint a MinMax
módszernél.  Kiegészítés: A minimax algoritmus alkalmazásának egyik nehézsége,
hogy egy játékhoz kétfajta heurisztikát is ki kell találni. Márpedig könnyű
észre venni, hogy a két heurisztika között erős összefüggés van. Az összefüggés
egyszerű szavakkal így írható le: A játék egy állapota az egyik játékosnak
minél jobb, a másiknak annál rosszabb. Azaz az egyik játékos heurisztikája
nyugodtan lehet a másik játékos heurisztikájának ellentéte (negáltja), vagyis
$-1$-szerese. Tehát, elég lesz 1 db heurisztika.

\subsection{Alfa-béta nyesés}

A gyakorlatban a MinMax algoritmus javított változatát szokás használni, az
alfa-béta nyesést.

\begin{definicio}
    Az ajánlott lépés.  A $b \in \mathcal{B}$ állásban {\bf ajánlott lépés} a
    $j \in \mathcal{J}$
    (lépni következő) játékos számára olyan $l \in \mathcal{L}$ lépés, amely esetén
    \begin{itemize}
        \item $b \in \dom l$,
        \item $o_l\left(\left<b,j \right> \right) = \left<b', j' \right>$, és
        \item $f(b,j,j,k) = f(b', j', j, k-1)$.
    \end{itemize}
\end{definicio}

A nyesés vagy metszés lehetővé teszi számunkra, hogy figyelmen kívül hagyjuk a
keresési fa azon részeit, amelyek nincsenek befolyással a végső választásra.

A heurisztikus kiértékelő függvények lehetővé teszik, hogy kimerítő keresés
nélkül meg tudjuk becsülni egy adott állapot valódi hasznosságát.  Tehát a
játékfa nagyobb részét is kihagyhatjuk adott esetben.

A konkrét vizsgált technika az alfa-béta nyesés (alpha-beta pruning). Ha ezt
egy standard minimax fára alkalmazzuk, ugyanazt az eredményt adja vissza, mint
a minimax, a döntésre hatással nem lévő ágakat azonban lenyesi.  A célunk az,
hogy meg tudjuk határozni a minimax döntést anélkül, hogy megkellene
vizsgálnunk az összes levélcsomópontot.

\begin{itemize}
    \item $\alpha =$ az út mentén tetszőleges döntési pontban a MAX számára eddig
        megtalált legjobb (azaz a legmagasabb értékű) választás értéke.
    \item $\beta =$ az út mentén tetszőleges döntési pontban a MIN számára eddig
        megtalált legjobb (azaz a legkisebb értékű) választás értéke.
\end{itemize}
