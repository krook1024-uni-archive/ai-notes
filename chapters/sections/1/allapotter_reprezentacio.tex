\section{Állapottér reprezentáció}

\subsection{Az állapottér fogalma}

Az állapottér-reprezentáció az egyik leggyakrabban használt reprezentációs mód
ami egy probléma formális megadására szolgál.

\begin{definicio}
    Probléma.
    Egy \textbf{probléma} (problem) formális megragadásához az alábbi négy
    komponensre van szükség:

    \begin{itemize}
        \item \textbf{kiinduló állapot}:
            amiből az ágens kezdi a cselekvéseit,
        \item \textbf{cselekvések halmaza}:
            ágens rendelkezésére álló lehetséges cselekvések,
        \item \textbf{állapotátmenet-függvény}:
            visszaadja a rendezett $\langle$ cselekvés, utódállapot $\rangle$
            párok halmazát (Egy alternatív megfogalmazás az \textbf{operátor}ok
            egy halmaza, amelyeket egy állapotra alkalmazva lehet az
            utódállapotokat generálni.) lásd még:
            \textit{\ref{operator-alkalmazasi}. Definíció}
        \item \textbf{állapottér}: A kezdeti állapot és az
            állapotátmenet-függvény együttesen implicit módon definiálják a
            probléma \textbf{állapotteré}t: azon állapotok halmazát, amelyek a
            kiinduló állapotból elérhetők.
    \end{itemize}
\end{definicio}

\begin{definicio}\label{operator-alkalmazasi}
    Operátor alkalmazási előfeltétel teszt.
    Ahhoz, hogy meggyőződjünk arról, hogy egy adott állapotra egy adott
    operátor alkalmazható, előbb meg kell vizsgálnunk, hogy az állapotra
    alkalmazható-e az operátor. Ezt a vizsgálódást operátor alkalmazási
    előfeltétel tesztnek nevezzük.
\end{definicio}

\begin{definicio}
    Célteszt.
    A célteszt ellenőrzi a célfeltételek teljesülését egy adott állapotban.
\end{definicio}

\subsection{Az állapottérgráf}

\begin{definicio}
    Állapottérgráf.
    Az állapottér egy gráfot alkot,
    amelynek csomópontjai az állapotok és a csomópontok közötti élek a
    cselekvések.


\end{definicio}

\begin{definicio}
    Állapottér útja.
    Az \textbf{állapottér egy útja} az állapotok egy sorozata, amely
    állapotokat a cselekvések egy sorozata köt össze.
\end{definicio}

\begin{definicio}
    Állapottér-reprezentációs gráf bonyolultsága.  Egy állapottér-reprezentált
    probléma megoldásának sikerét jelentősen befolyásolja a reprezentációs gráf
    bonyolultsága:

    \begin{itemize}
        \item a csúcsok száma,
        \item az egy csúcsból kiinduló élek száma,
        \item a hurkok és körök száma és hossza.
    \end{itemize}

    Ezért célszerű minden lehetséges egyszerűsítést végrehajtani. Lehetséges
    egyszerűsítések:

    \begin{itemize}
        \item a csúcsok számának csökkentése — ügyes reprezentációval az
            állapottér kisebb méretű lehet;
        \item az egy csúcsból kiinduló élek számának csökkentése — az
            operátorok értelmezési tartományának alkalmas megválasztásával
            érhető el;
        \item a reprezentációs gráf fává alakítása — a hurkokat, illetve
            köröket „kiegyenesítjük”
    \end{itemize}
\end{definicio}

\subsubsection{Az állapottérgráf jellemzése}

$b$: Az {\bf elágazási tényező} (branching factor) a tetszőleges
állapotból közvetlenül elérhető állapotok maximális száma. \[
\max \{\card \{b:a \Rightarrow b\} : a \in \mathcal{A})\}
.\]

$d$ : A {\bf legsekélyebb megoldás} a legkevesebb operátoralkalmazás
segítségével elérhető célállapot eléréshez szükséges operátoralkalmazások száma.
(A legrövidebb megoldás hossza. A célállapotok minimális mélysége.)
A legkisebb olyan $i$ amely esetén van olyan állapotsorozat, hogy \[
    a_0 \Rightarrow a_1, \quad a_1 \Rightarrow a_2 \quad \ldots \quad a_{i-1}
    \Rightarrow a_i,\quad a_i \in \mathcal{C}
.\]

$m$ : A {\bf csomópontok maximális mélysége}. A legnagyobb olyan $i$, amely esetén
van olyan állapotsorozat, hogy \[
    a_0 \Rightarrow a_1,\quad a_1 \Rightarrow a_2 \quad \ldots \quad a_{i-1}
    \Rightarrow a_i
.\]

% TODO

\subsection{Költség és heurisztika fogalmak}

\begin{definicio}
    Lépésköltség.
    Az $x$ állapotból az $y$ állapotba vezető $cs$ cselekvés
    \textbf{lépésköltség}e (step cost) legyen $lk(x, cs, y)$.  Tételezzük fel,
    hogy a lépésköltség nemnegatív.
\end{definicio}

\begin{definicio}
    Útköltség-függvény.
    Egy \textbf{útköltség-függvény}, egy olyan függvény amely az állapottér
    minden útjához hozzárendel egy költséget.
\end{definicio}

\begin{definicio}
    Megoldás.
    Egy út, amely a kiinduló állapotból egy célállapotba vezet.
\end{definicio}

\begin{definicio}
    Optimális megoldás.
    A legkisebb útköltségű megoldás.
\end{definicio}

% TODO
