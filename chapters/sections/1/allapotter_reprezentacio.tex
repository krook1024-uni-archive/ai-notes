\section{Állapottér reprezentáció}

\subsection{Az állapottér fogalma}

Az állapottér-reprezentáció az egyik leggyakrabban használt reprezentációs mód
ami egy probléma formális megadására szolgál.

\begin{definicio}
    Probléma.
    Egy \textbf{probléma} (problem) formális megragadásához az alábbi négy
    komponensre van szükség:

    \begin{itemize}
        \item \textbf{kiinduló állapot}:
            amiből az ágens kezdi a cselekvéseit,
        \item \textbf{cselekvések halmaza}:
            ágens rendelkezésére álló lehetséges cselekvések,
        \item \textbf{állapotátmenet-függvény}:
            visszaadja a rendezett $\langle$ cselekvés, utódállapot $\rangle$
            párok halmazát (Egy alternatív megfogalmazás az \textbf{operátor}ok
            egy halmaza, amelyeket egy állapotra alkalmazva lehet az
            utódállapotokat generálni.) lásd még:
            \textit{\ref{operator-alkalmazasi}. Definíció}
        \item \textbf{állapottér}: A kezdeti állapot és az
            állapotátmenet-függvény együttesen implicit módon definiálják a
            probléma \textbf{állapotteré}t: azon állapotok halmazát, amelyek a
            kiinduló állapotból elérhetők.
    \end{itemize}
\end{definicio}

\begin{definicio}\label{operator-alkalmazasi}
    Operátor alkalmazási előfeltétel teszt.
    Ahhoz, hogy meggyőződjünk arról, hogy egy adott állapotra egy adott
    operátor alkalmazható, előbb meg kell vizsgálnunk, hogy az állapotra
    alkalmazható-e az operátor. Ezt a vizsgálódást operátor alkalmazási
    előfeltétel tesztnek nevezzük.
\end{definicio}

\begin{definicio}
    Célteszt.
    A célteszt ellenőrzi a célfeltételek teljesülését egy adott állapotban.
\end{definicio}

\begin{definicio}
    Állapottér-reprezentáció.
    Állapottér-reprezentáció alatt egy formális négyest értünk: \[
        \left< \mathcal{A}, k, \mathcal{C}, \mathcal{O} \right>
    ,\] ahol \begin{itemize}
    \item $\mathcal{A}$: az állapotok halmaza, $\mathcal{A} \neq \varnothing$
    \item $k \in \mathcal{A}$: a kezdőállapot
    \item $\mathcal{C} \subset \mathcal{A}$: a célállapotok halmaza
    \item $\mathcal{O}$: az operátorok halmaza, ahol $\mathcal{O} \neq \varnothing$
        \begin{itemize}
            \item $\forall o \in \mathcal{O}$ egy operátor, azaz egy $o :
                \text{Dom}(o) \to \mathcal{A}$ függvény, ahol $\text{Dom}(o) =
                \{a | a \not\in C \land \text{előfeltétel}(o,a)\}$
        \end{itemize}
    \end{itemize}
\end{definicio}

\begin{definicio}
    Közvetlen elérhetőség.
    Az $a \in \mathcal{A}$ állapotból az $a' \in \mathcal{A}$ állapot
    \textbf{közvetlenül elérhető}, ha van olyan $o_l \in \mathcal{O}$,
    amely esetén \[
        a \in \dom(o_l) \text{  és  } o_l(a) = a'
    \]
    és ezt $a \Rightarrow a'$ alakban jelöljük.
\end{definicio}

\begin{definicio}
    Elérhetőség.
    Az $a \in \mathcal{A}$ állapotból az $a' \in \mathcal{A}$ állapot
    \textbf{elérhető} ($a \Rightarrow^* a'$), ha
    \begin{itemize}
        \item $a=a'$, vagy
        \item van olyan  $a_1, a_2, \ldots, a_k$ állapotsorozat, hogy
            $a_1=a, a_k=a'$ továbbá \[
                a_i \Rightarrow a_{i+1} \text{ minden }
                i \in \{1, \ldots, k-1\} \text{ esetén}
            .\]
    \end{itemize}
\end{definicio}

\subsection{Az állapottérgráf}

\begin{definicio}
    Állapottérgráf.
    Az állapottér egy gráfot alkot,
    amelynek csomópontjai az állapotok és a csomópontok közötti élek a
    cselekvések.
\end{definicio}

\begin{definicio}
    Állapottér útja.
    Az \textbf{állapottér egy útja} az állapotok egy sorozata, amely
    állapotokat a cselekvések egy sorozata köt össze.
\end{definicio}

\begin{definicio}
    Állapottér-reprezentációs gráf bonyolultsága.  Egy állapottér-reprezentált
    probléma megoldásának sikerét jelentősen befolyásolja a reprezentációs gráf
    bonyolultsága:

    \begin{itemize}
        \item a csúcsok száma,
        \item az egy csúcsból kiinduló élek száma,
        \item a hurkok és körök száma és hossza.
    \end{itemize}

    Ezért célszerű minden lehetséges egyszerűsítést végrehajtani. Lehetséges
    egyszerűsítések:

    \begin{itemize}
        \item a csúcsok számának csökkentése — ügyes reprezentációval az
            állapottér kisebb méretű lehet;
        \item az egy csúcsból kiinduló élek számának csökkentése — az
            operátorok értelmezési tartományának alkalmas megválasztásával
            érhető el;
        \item a reprezentációs gráf fává alakítása — a hurkokat, illetve
            köröket „kiegyenesítjük”
    \end{itemize}
\end{definicio}

\subsubsection{Az állapottérgráf jellemzése}

$b$: Az {\bf elágazási tényező} (branching factor) a tetszőleges
állapotból közvetlenül elérhető állapotok maximális száma. \[
\max \{\card \{b:a \Rightarrow b\} : a \in \mathcal{A})\}
.\]

$d$ : A {\bf legsekélyebb megoldás} a legkevesebb operátoralkalmazás
segítségével elérhető célállapot eléréshez szükséges operátoralkalmazások száma.
(A legrövidebb megoldás hossza. A célállapotok minimális mélysége.)
A legkisebb olyan $i$ amely esetén van olyan állapotsorozat, hogy \[
    a_0 \Rightarrow a_1, \quad a_1 \Rightarrow a_2 \quad \ldots \quad a_{i-1}
    \Rightarrow a_i,\quad a_i \in \mathcal{C}
.\]

$m$ : A {\bf csomópontok maximális mélysége}. A legnagyobb olyan $i$, amely esetén
van olyan állapotsorozat, hogy \[
    a_0 \Rightarrow a_1,\quad a_1 \Rightarrow a_2 \quad \ldots \quad a_{i-1}
    \Rightarrow a_i
.\]

% TODO

\subsection{Költség és heurisztika fogalmak}

\begin{definicio}
    Lépésköltség.
    Az $x$ állapotból az $y$ állapotba vezető $cs$ cselekvés
    \textbf{lépésköltség}e (step cost) legyen $lk(x, cs, y)$.  Tételezzük fel,
    hogy a lépésköltség nemnegatív.
\end{definicio}

\begin{definicio}
    Útköltség-függvény.
    Egy \textbf{útköltség-függvény}, egy olyan függvény amely az állapottér
    minden útjához hozzárendel egy költséget.
\end{definicio}

\begin{definicio}
    Megoldás.
    Egy út, amely a kiinduló állapotból egy célállapotba vezet.
\end{definicio}

\begin{definicio}
    Optimális megoldás.
    A legkisebb útköltségű megoldás.
\end{definicio}

\begin{definicio}
    Heurisztikus függvény.
    Egy $\left<\mathcal{A}, k, \mathcal{C}, \mathcal{O} \right>$
    állapottér-reprezentációhoz megadott heurisztika egy olyan \[
        h : \mathcal{A} \mapsto \mathbb{N} \text{ függvény, melyre }
        \forall c \in \mathcal{C} h(c) = 0
    .\]

    A heurisztika nem más mint egy becslés amely megmondja egy csúcsra nézve,
    hogy melyik gyermeke felé induljon tovább a keresésben. A heurisztikus
    kereső nem teljesen megbízható mivel a részfák még nincsenek legenerálva,
    azaz nem lehetünk biztosak benne.

    Egyes heurisztikákat perfekt heurisztikának nevezünk és egy ilyen
    heurisztika legfeljebb $1+n*d$ csúcsot generál le, azaz ilyen heurisztika
    eseté a csúcsok száma már csak lineáris függvénye a megoldás hosszának.(
    Perfekt heurisztika egyébként nem létezik, mert ha lenne akkor már előre
    ismernénk a megoldást - nem lenne értelme keresni.)
\end{definicio}
