\section{Ágens szemlélet}

\subsection{Az ágens fogalma}

\begin{definicio}
    Ágens.
    Egy \textbf{ágens} (agent) bármi lehet, amit úgy tekinthetünk, mint ami az
    \textbf{érzékelői} (sensors) segítségével érzékeli a \textbf{környezetét}
    (environment), és \textbf{beavatkozói} (actuators) segítségével
    megváltoztatja azt.

    Az emberi ágensnek van szeme, füle és egyéb szervei az érzékelésre, és keze,
    lába, szája és egyéb testrészei a beavatkozásra.

    A robotágens kamerákat és infravörös távolsági keresőket használ
    érzékelőként, és különféle motorokat beavatkozóként.

    A szoftverágens billentyűleütéseket, fájltartalmakat és hálózati
    adatcsomagokat fogad érzékelőinek bemeneteként, és képernyőn történő
    kijelzéssel, fájlok írásával, hálózati csomagok küldésével avatkozik be a
    környezetébe.

    Azzal az általános feltételezéssel fogunk élni, hogy minden ágens képes
    saját akcióinak érzékelésére (de nem mindig látja azok hatását).
\end{definicio}

\begin{definicio}
    Érzékelés.
    Az érzékelés (percept) fogalmat használjuk az ágens érzékelő bemeneteinek
    leírására egy tetszőleges pillanatban.
\end{definicio}

\begin{definicio}
    Ágens érzékelési sorozata.
    Egy ágens érzékelési sorozata (percept sequence) az ágens érzékeléseinek
    teljes története, minden, amit az ágens valaha is érzékelt. Általánosságban,
    egy adott pillanatban egy ágens cselekvése az addig megfigyelt teljes
    érzékelési sorozatától függhet. Ha az összes lehetséges érzékelési
    sorozathoz meg tudjuk határozni az ágens lehetséges cselekvéseit, akkor
    lényegében mindent elmondtunk az ágensről.
\end{definicio}

\begin{definicio}
    Ágensfüggvény.
    Matematikailag megfogalmazva azt mondhatjuk, hogy az ágens viselkedését az
    ágensfüggvény (agent function) írja le, ami az adott érzékelési sorozatot
    egy cselekvésre képezi le.
\end{definicio}

\begin{definicio}
    Ágensprogram.
    Egy mesterséges ágens belsejében az ágensfüggvényt egy ágensprogram (agent
    program) valósítja meg.
\end{definicio}

\begin{megjegyzes}
    Különbség ágensfüggvény és ágensprogram között.
    Az ágensfüggvény egy absztrakt matematikai leírás, az ágensprogram egy
    konkrét implementáció, amely az ágens architektúráján működik.
\end{megjegyzes}

\subsection{Az ágens jellemzése (teljesítmény, környezet, érzékelők, beavatkozók}

\begin{definicio}
    \textbf{TKBÉ} (PEAS) leírás.
    Egy ágens tervezése során az első lépésnek mindig a feladatkörnyezet lehető
    legteljesebb meghatározásának kell lennie.

            \begin{tabularx}{\textwidth}{|X|X|X|X|X|}
            \hline
            \textbf{Ágenstípus} &
            \textbf{Tel-jesítmény-mérték (Performance)} &
            \textbf{Környezet (Environment)} &
            \textbf{Beavatkozók (Actuators)} &
            \textbf{Érzékelők (Sensors)}
            \\\hline
            Taxisofőr &
            Biztonságos, gyors, törvényes, kényelmes utazás, maximum haszon &
            Utak, egyéb forgalom, gyalogosok, ügyfelek &
            Kormány, gáz, fék, index, kürt, kijelző &
            Kamerák, hangradas, sebességmérő, GPS, kilométeróra, motorérzékelők,
            billentyűzet
            \\\hline
        \end{tabularx}
\end{definicio}

\begin{definicio}
    Teljesen megfigyelhető környezet.

    Ha az ágens szenzorai minden pillanatban hozzáférést nyújtanak a környezet
    teljes állapotához, akkor azt mondjuk, hogy a környezet teljesen
    megfigyelhető.
\end{definicio}

\begin{definicio}
    Determinisztikus (deterministic) vagy sztochasztikus (stochastic).

    Amennyiben a környezet következő állapotát jelenlegi állapota és az ágens
    által végrehajtott cselekvés teljesen meghatározza, akkor azt mondjuk, hogy
    a környezet determinisztikus, egyébként sztochasztikus.
\end{definicio}

\begin{definicio}
    Epizódszerű (episodic) vagy sorozatszerű (sequential) környezet.

    Epizódszerű környezetben az ágens tapasztalata elemi "epizódokra" bontható.
    Minden egyes epizód az ágens észleléseiből és egy cselekvéséből áll. Nagyon
    fontos, hogy a következő epizód nem függ az előzőben végrehajtott
    cselekvésektől.  Epizódszerű környezetekben az egyes epizódokban az akció
    kiválasztása csak az aktuális epizódtól függ.
\end{definicio}

\begin{definicio}
    Statikus (static) vagy dinamikus (dynamic) környezet.

    Ha a környezet megváltozhat, amíg az ágens gondolkodik, akkor azt mondjuk,
    hogy a környezet az ágens számára dinamikus; egyébként statikus.
\end{definicio}

\begin{definicio}
    Diszkrét (discrete) vagy folytonos (continuous) környezet.

    A diszkrét/folytonos felosztás alkalmazható a környezet állapotára, az
    időkezelés módjára, az ágens észleléseire, valamint cselekvéseire. Például
    egy diszkrét állapotú környezet, mint amilyen a sakkjáték, véges számú
    különálló állapottal rendelkezik. A sakkban szintén diszkrét az akciók és
    cselekvések halmaza. A taxivezetés folytonos állapotú és idejű probléma: a
    sebesség, a taxi és más járművek helye folytonos értékek egy tartományát
    járja végig a folytonos időben.
\end{definicio}

\begin{definicio}
    Egyágenses (single agent) vagy többágenses (multiagent) környezet.

    Az egyágenses és többágenses környezetek közötti különbségtétel egyszerűnek
    tűnhet. Például a keresztrejtvényt megfejtő ágens önmagában nyilvánvalóan
    egyágenses környezetben van, míg egy sakkozó ágens egy kétágensesben.
    Vannak azonban kényes kérdések. Először is: leírtuk azt, hogy egy entitás
    hogyan tekinthető ágensnek, ugyanakkor nem magyaráztuk meg, mely entitások
    tekintendők ágensnek. Egy $A$ ágensnek (például a taxisofőrnek) egy $B$
    objektumot (egy másik járművet) ágensnek kell tekintenie, vagy egyszerűen
    egy sztochasztikusan viselkedő dolognak, a tengerparti hullámokhoz vagy a
    szélben szálló falevelekhez hasonlatosan? A választás kulcsa az, hogy vajon
    $B$ viselkedése legjobban egy $A$ viselkedésétől függő teljesítménymérték
    maximalizálásával írhatóe le. Például a sakkban a $B$ ellenfél saját
    teljesítménymértékét próbálja maximalizálni, amely – a sakk szabályainak
    következtében – A teljesítménymértékét minimalizálja. Így a sakk egy
    versengő (\textbf{competitive}) többágenses környezet. Másrészről, a taxi
    vezetési környezetben az ütközések elkerülése az összes ágens
    teljesítménymértékét maximálja, így az részben kooperatív
    (\textbf{cooperative}) többágenses környezet.  Emellett részben versengő
    is, hiszen például csak egy autó tud egy parkolóhelyet elfoglalni. A
    többágenses környezetekben felmerülő ágenstervezési problémák gyakran
    egészen mások, mint egyágenses környezetekben. Többágenses környezetekben
    például a kommunikáció (communication) gyakran racionális viselkedésként
    bukkan fel; egyes részlegesen megfigyelhető versengő környezetekben a
    sztochasztikus viselkedés racionális, hiszen így elkerülhetők a
    megjósolhatóság csapdái.
\end{definicio}
